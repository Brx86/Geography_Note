\PassOptionsToPackage{quiet}{xeCJK} %抑制警告 Font "FandolSong-Regular" does not contain requested(fontspec)	Script "CJK".
\documentclass[a4paper]{article}
\usepackage[UTF8]{ctex}
\usepackage{fontspec} %添加标题用的
\usepackage{makecell} %表格内换行用的
%\usepackage{hyperref} %生成书签目录
\usepackage{pdfcomment} %生成书签目录
%上面那两个差不多
\usepackage{enumitem} %解决LaTeX列表最大深度问题
\setlistdepth{9} %设置列表的最大深度

\setlist[itemize,1]{label=$\bullet$}
\setlist[itemize,2]{label=$\bullet$}
\setlist[itemize,3]{label=$\bullet$}
\setlist[itemize,4]{label=$\bullet$}
\setlist[itemize,5]{label=$\bullet$}
\setlist[itemize,6]{label=$\bullet$}
\setlist[itemize,7]{label=$\bullet$}
\setlist[itemize,8]{label=$\bullet$}
\setlist[itemize,9]{label=$\bullet$}

\renewlist{itemize}{itemize}{9}

\title{地理高考复习提纲}
\author{孙文彬}
\date{\today}

\begin{document}
    \maketitle%生成标题
    \newpage
    \tableofcontents%生成目录
    \newpage
    \section{答题模板}
    \subsection{水循环-蒸发的影响因素}
    \begin{enumerate}
        \item 光照
        \item 气温
        \item 风速
        \item 湿度
        \item 裸露的水域表面积
    \end{enumerate}
    \subsection{水汽输送的影响因素}
    \begin{enumerate}
        \item 风带(信风带,西风带)
        \item 海陆热力性质差异形成的季风
        \item 距海远近,是否有地形阻挡
    \end{enumerate}
    \subsection{水循环-地表径流的影响因素}
    \begin{enumerate}
        \item 年降水量
        \item 流域面积(支流状况)
        \item 植被
        \item 地质条件(土壤质地)
        \item 蒸发
        \item 人类活动取水
    \end{enumerate}
    \subsection{下渗的影响因素}
    \begin{enumerate}
        \item 地面性质
        \item 坡度
        \item 植被
        \item 降水强度
        \item 降水持续时间
    \end{enumerate}
    \subsection{内流区的形成条件}
    \begin{enumerate}
        \item 地形
            \begin{enumerate}
                \item 地势低洼,或者地形封闭(河流不能流入海洋)
            \end{enumerate}
        \item 气候降水稀少,蒸发旺盛(河流水量少,难以流入海洋)
        \item 海陆位置
            \begin{enumerate}
                \item 内陆地区远离海洋,降水少,距离海洋远
            \end{enumerate}
    \end{enumerate}
    \subsection{航运价值}
    \begin{enumerate}
        \item 自然条件
        \begin{enumerate}
            \item 地形
            \begin{enumerate}
                \item 水速
            \end{enumerate}
            \item 气候
            \begin{enumerate}
                \item 水量,水位变化,结冰期
            \end{enumerate}
            \item 植被
            \begin{enumerate}
                \item 含沙量
            \end{enumerate}
            \item 水系
            \begin{enumerate}
                \item 河流长度,支流,流域面积
            \end{enumerate}
        \end{enumerate}
        \item 社会经济条件
        \begin{enumerate}
            \item 流域内的经济,人口,交通网络,经济腹地
        \end{enumerate}
    \end{enumerate}
    \subsection{河流的水文特征}
    \begin{enumerate}
        \item 流量
        \begin{enumerate}
            \item 大小,季节变化,有无断流(降水特征,雨水补给河流面积大小)
        \end{enumerate}
        \item 含沙量
        \begin{enumerate}
            \item 取决于流域的植被状况,地面物质结构,地面坡度,降水强度
        \end{enumerate}
        \item 结冰期
        \begin{enumerate}
            \item 有无,长短,有无凌汛
        \end{enumerate}
        \item 流速(与地势起伏大小有关)
        \item 汛期长短
        \begin{enumerate}
            \item 高低,变化特征(河流补给类型,水利工程,湖泊调蓄作用)
        \end{enumerate}
        \item 水能
        \begin{enumerate}
            \item 地形(河流落差大小,流速快慢)
            \item 气候(降水量的多少,径流量的大小,蒸发量的大小)
        \end{enumerate}
    \end{enumerate}
    \subsection{水系特征}
    \begin{enumerate}
        \item 长度
        \item 流向
        \item 流域面积大小
        \item 落差大小(水能)
        \item 河道曲直情况
        \item 支流多少
        \item 河流支流排列形状(扇形,树枝状)
    \end{enumerate}
    \subsection{流量的影响因素}
    \begin{enumerate}
        \item 补给(以冰川融水作为补给来源的河流,一般径流量小;降水看气候类型)
        \item 流域面积
        \item 支流数量
        \item 下渗与蒸发(土壤类型)
        \item 湖泊,沼泽等湿地的分布状况
        \item 植被覆盖
        \item 人类活动(工农业,水库,植树造林)
    \end{enumerate}
    \subsection{含沙量的影响因素}
    \begin{enumerate}
        \item 流域的植被状况
        \item 土质(疏松,黏实)
        \item 地形起伏和流速(地势高,坡度陡:以侵蚀作用为主,含沙量大)
        \item 降水强度与降水量
        \item 人类活动
    \end{enumerate}
    \subsection{气候特征描述}
    \begin{enumerate}
        \item 气温特征
        \begin{enumerate}
            \item 气温的高低(最高温,最低温,均温)
            \item 气温的日较差
            \item 气温的年较差
        \end{enumerate}
        \item 降水的特征
        \begin{enumerate}
            \item 降水总量
            \item 降水季节性变化
            \item 降水年际变化
            \item 雨季时间,长短
        \end{enumerate}
    \end{enumerate}
    \subsection{气候的影响因素}
    \begin{enumerate}
        \item 纬度
        \item 大气环流
        \item 地形
        \item 海陆位置
        \item 洋流
        \item 人类活动(热岛效应,温室效应)
    \end{enumerate}
    \subsection{气温的影响因素}
    \begin{enumerate}
        \item 纬度(决定因素)
        \begin{enumerate}
            \item 太阳高度,昼长
        \end{enumerate}
        \item 大气环流
        \item 地形(海拔,地势)
        \begin{enumerate}
            \item 谷地热量不易散失
            \item 山脉对风的阻挡
        \end{enumerate}
        \item 海陆位置
        \item 洋流
        \item 天气状况
        \item 下垫面
        \item 人类活动,热岛效应,温室效应
    \end{enumerate}
    \subsection{降水的影响因素}
    \begin{enumerate}
        \item 大气环流
        \item 地形
        \item 地势
        \item 海陆位置
        \item 洋流
        \item 下垫面
        \item 人类活动,热岛效应,温室效应
    \end{enumerate}
    \subsection{太阳辐射的影响因素}
    \begin{enumerate}
        \item 纬度(太阳高度角)
        \item 日照时间
        \item 天气状况
        \item 地势高低
        \item 太阳透明度
    \end{enumerate}
    \subsection{昼夜温差的影响因素}
    \begin{enumerate}
        \item 地势
        \item 天气状况
        \item 下垫面性质
        \item 纬度(太阳高度角)
    \end{enumerate}
    \subsection{气候对农业的影响}
    \begin{enumerate}
        \item 光照
        \item 水分
        \item 热量
        \item 水热配合
        \item 昼夜温差
        \item 气象灾害
    \end{enumerate}
    \subsection{风能开发的区位条件}
    \begin{enumerate}
        \item 自然条件(风力大)
        \begin{enumerate}
            \item 气压梯度力
            \item 摩擦力(高原,平原等地面平坦开阔,水域,植被多少)
            \item 距高压中心的远近(季风源地)
            \item 特殊地形(山谷的狭管效应)
        \end{enumerate}
        \item 社会经济条件
        \begin{enumerate}
            \item 市场大小,远近
            \item 资金,技术
            \item 建设用地
            \item 政策
            \item 风电优点(清洁能源,可再生)
        \end{enumerate}
    \end{enumerate}
    \subsection{水能开发的条件}
    \begin{enumerate}
        \item 自然条件
        \begin{enumerate}
            \item 水能丰富(流量,落差)
            \item 建坝条件
            \begin{enumerate}
                \item 峡谷多,蓄水量大,利于建坝
            \end{enumerate}
            \begin{enumerate}
                \item 地质稳定,没有断层裂隙溶洞
            \end{enumerate}
        \end{enumerate}
        \item 社会经济条件
        \begin{enumerate}
            \item 淹没面积小,人口少,损失小
            \item 输电距离
            \item 技术
            \item 市场
            \item 生态效益清洁能源
        \end{enumerate}
    \end{enumerate}
    \subsection{渔场的形成条件}
    \begin{enumerate}
        \item 地形
        \begin{enumerate}
            \item 面积广阔的大陆架,阳光直射,光合作用强,饵料丰富
        \end{enumerate}
        \item 温带海域
        \begin{enumerate}
            \item 气温变化大,海水上泛
        \end{enumerate}
        \item 河口处
        \begin{enumerate}
            \item 河流带来丰富的营养盐类
        \end{enumerate}
        \item 洋流(交汇流或上升流)
        \begin{enumerate}
            \item 海水上泛,带来海底营养盐类,饵料丰富
        \end{enumerate}
        \item 中低纬度
        \begin{enumerate}
            \item 生物生长快
        \end{enumerate}
    \end{enumerate}
    \subsection{海水盐度的影响因素}
    \begin{enumerate}
        \item 气候因素(主要因素)
        \begin{enumerate}
            \item 降水量与蒸发量的关系
        \end{enumerate}
        \item 洋流因素
        \begin{enumerate}
            \item 同一纬度海区,有暖流经过盐度偏高
            \item 有寒流经过盐度偏低
        \end{enumerate}
        \item 河流径流注入因素
        \begin{enumerate}
            \item 有大量河水汇入的海区,盐度偏低
        \end{enumerate}
        \item 海区的封闭度
        \begin{enumerate}
            \item 即与附近海区海水的交换量
        \end{enumerate}
        \item 高纬度结,融冰量的大小
        \begin{enumerate}
            \item 有结冰现象的海区,盐度偏高
            \item 有融冰现象的海区,盐度偏低
        \end{enumerate}
    \end{enumerate}
    \subsection{水体结冰的影响因素}
    \begin{enumerate}
        \item 气温
        \begin{enumerate}
            \item 纬度
            \item 地形
            \item 洋流
            \item 海陆位置
            \item 人类活动
        \end{enumerate}
        \item 水文特征
        \begin{enumerate}
            \item 水量大,水深,面积大,不易结冰
            \item 含盐量大,不易结冰
            \item 流速快,不易结冰
        \end{enumerate}
        \item 地热
        \begin{enumerate}
            \item 温泉
        \end{enumerate}
        \item 水域分布密度
    \end{enumerate}
    \subsection{雪线高度的影响因素}
    \begin{enumerate}
        \item 降水(当地气候特征情况迎风坡降水多,雪线低)
        \item 气温(阳坡雪线高于阴坡,不同纬度的温度变化,0℃等温线的海拔的高低)
    \end{enumerate}
    \subsection{山地垂直带谱的影响因素}
    \begin{enumerate}
        \item 纬度
        \begin{enumerate}
            \item 山地所处的纬度越高,带谱越简单
        \end{enumerate}
        \item 海拔
        \begin{enumerate}
            \item 山地的海拔越高,带谱可能越复杂
        \end{enumerate}
        \item 热量(即阳坡,阴坡)
        \begin{enumerate}
            \item 影响同一带谱的海拔高度
        \end{enumerate}
        \item 降水(迎风坡,背风坡)
    \end{enumerate}
    \subsection{生物多样性的影响因素}
    \begin{enumerate}
        \item 水分和热量
        \begin{enumerate}
            \item 高温多雨,水热条件优越的地区,生物种类丰富
        \end{enumerate}
        \item 自然环境
        \begin{enumerate}
            \item 水分热量复杂,地形复杂,气候复杂,垂直变化,水平变化(南北:热量,东西:水分)
        \end{enumerate}
        \item 环境变迁与突发事件
        \begin{enumerate}
            \item 如:地质时期的冰期,陨石撞击地球,全球变暖,臭氧层破坏环境
        \end{enumerate}
        \item 天敌和外来物种
        \item 人类活动的破坏和干扰
        \begin{enumerate}
            \item 滥捕滥猎,食物链破坏,对动物栖息地的破坏
            \item 动物食用被污染的食物等
        \end{enumerate}
    \end{enumerate}
    \subsection{土壤肥力高低的影响因素}
    \begin{enumerate}
        \item 流失
        \begin{enumerate}
            \item 微生物分解
            \item 淋溶作用
            \item 植物吸收
        \end{enumerate}
        \item 来源
        \begin{enumerate}
            \item 植物残体
        \end{enumerate}
    \end{enumerate}
    \subsection{成土母质对土壤的影响}
    \begin{enumerate}
        \item 土壤形成的物质基础(矿物质)
        \item 植物的矿物养分来源
        \item 决定土壤的颗粒度的大小
        \item 决定土壤中的化学元素和养分
    \end{enumerate}
    \subsection{气候对土壤的影响}
    \begin{enumerate}
        \item 直接影响
        \begin{enumerate}
            \item 土壤的水热状况,形成不同类型的土壤
        \end{enumerate}
        \item 影响土壤有机质含量
        \begin{enumerate}
            \item 湿热气候,微生物分解旺盛,淋溶作用强,有机质少(寒冷气候反之)
        \end{enumerate}
        \item 影响风化壳形成过程
        \begin{enumerate}
            \item 降水多,气温高,风化壳厚,土壤越厚
        \end{enumerate}
    \end{enumerate}
    \subsection{土壤盐碱化的成因}
    \begin{enumerate}
        \item 自然原因
            \item 气候(气温与降水)
            \begin{enumerate}
                \item 干旱,降水少
                \item 气温高,多大风天气,蒸发旺盛
            \end{enumerate}
            \item 地形
            \begin{enumerate}
                \item 地势低洼,排水不畅,地表水下渗,地下水位升高
            \end{enumerate}
            \item 地下水
            \begin{enumerate}
                \item 地下水位埋藏浅
            \end{enumerate}
            \item 土壤
            \begin{enumerate}
                \item 碱性土壤
            \end{enumerate}
        \item 人为原因
            \item 不合理的灌溉
            \begin{enumerate}
                \item 大水漫灌,只灌不排,导致水在地表聚集,大量下渗
                \item 地下水位升高,盐分被带到地表
            \end{enumerate}
            \item 沿海地区过度抽取地下水
            \begin{enumerate}
                \item 引起海水入侵地下水,进而随着地下水上升,增加了土地盐分
            \end{enumerate}
            \item 兴修水利工程
            \begin{enumerate}
                \item 补给地下水,水位升高
            \end{enumerate}
        \item 治理措施
        \begin{enumerate}
            \item 引淡淋盐
            \item 井排井灌,建立现代化排水系统
            \item 节水灌溉技术(滴灌喷灌)
            \item 地膜覆盖(抑制水分蒸发)
            \item 生物措施
            \begin{enumerate}
                \item 种植耐盐碱作物
            \end{enumerate}
        \end{enumerate}
    \end{enumerate}
    \subsection{航天发射基地的条件}
    \begin{enumerate}
        \item 纬度越低,地球自转线速度越大,节省燃料和成本
        \item 气象条件
        \begin{enumerate}
            \item 阴雨天气少,晴天多,湿度低,风速小
        \end{enumerate}
        \item 地形地质
        \begin{enumerate}
            \item 地势平坦开阔,地质结构稳定
        \end{enumerate}
        \item 交通条件
        \begin{enumerate}
            \item 具有良好的交通条件,要能够运输大型火箭
        \end{enumerate}
        \item 安全条件
        \begin{enumerate}
            \item 建于山区,沙漠地区,人烟稀少
        \end{enumerate}
    \end{enumerate}
    \subsection{回收基地的条件}
    \begin{enumerate}
        \item 人烟稀少
        \item 地势平坦开阔
        \item 无大河,大湖,大片森林
        \item 气象条件
    \end{enumerate}
    \subsection{天文观测基地的区位选择}
    \begin{enumerate}
        \item 自然条件
        \begin{enumerate}
            \item 地理位置
            \begin{enumerate}
                \item 纬度高,星空起落范围小,利于观测
                \item 纬度低,观测星空范围小
                \item 夜长,连续观测时间长
                \item 南半球观测南天区,北半球观测北天区
            \end{enumerate}
            \item 气象条件
            \begin{enumerate}
                \item 晴天多,云量小,大气透明度大
                \item 风速小,大气湍流少
            \end{enumerate}
            \item 地形条件
            \begin{enumerate}
                \item 地势高,空气稀薄,透明度好
                \item 海拔高,视野开阔
            \end{enumerate}
        \end{enumerate}
        \item 人文条件
        \begin{enumerate}
            \item 附近人口稀少(远离城市),光污染少
            \item 大气污染少
            \item 干扰少
            \item 技术设备
            \item 政策
        \end{enumerate}
    \end{enumerate}
    \subsection{流程的影响因素}
    \begin{enumerate}
        \item 陆地面积大小
        \item 大陆轮廓形态(完整或破碎)
        \item 分水岭影响
    \end{enumerate}
    \subsection{流域面积的影响因素}
    \begin{enumerate}
        \item 地形
        \item 气候(降水量和干湿状况)
        \item 陆域面积大小
        \item 河道状况
    \end{enumerate}
    \subsection{河网密度的影响因素}
    \begin{enumerate}
        \item 气候(降水量和干湿状况)
        \item 地形地势
        \item 植被
        \item 岩石土壤的渗透性和抗蚀能力
    \end{enumerate}
    \subsection{评价水利工程的影响}
    \begin{enumerate}
        \item 有利影响
        \begin{enumerate}
            \item 经济效益
            \begin{enumerate}
                \item 产生防洪,发电,航运,灌溉和旅游等综合经济效益
            \end{enumerate}
            \item 生态效益
            \begin{enumerate}
                \item 调节库区气候,缓解生态环境压力
                \item 拦截泥沙,降低河流含沙量,提高水质
            \end{enumerate}
            \item 社会效应
            \begin{enumerate}
                \item 降低洪涝威胁,保障人们生命财产安全
                \item 促进产业调整,推动库区经济发展
            \end{enumerate}
        \end{enumerate}
        \item 不利影响
        \begin{enumerate}
            \item 库区
            \begin{enumerate}
                \item 库区淤积泥沙,库容减小
                \item 易诱发地质灾害,如地震
                \item 影响生物洄游,生物多样性减少
                \item 淹没文物古迹
                \item 水质变差
                \item 库区蓄水后地下水位上升,易导致土地盐碱化
            \end{enumerate}
            \item 下游
            \begin{enumerate}
                \item 地貌
                \begin{enumerate}
                    \item 来水来沙减少,三角洲萎缩,海水入侵,海岸线后退
                \end{enumerate}
                \item 气候降水减少
                \item 河流径流量减少,海水倒灌,水质变差
                \item 土壤
                \begin{enumerate}
                    \item 海水倒灌,土壤盐碱化加剧,泥沙沉积减少
                    \item 土地肥力下降,农业减产
                \end{enumerate}
                \item 生物
                \begin{enumerate}
                    \item 河口渔业资源减少
                    \item 湿地减少,生物多样性减少
                \end{enumerate}
            \end{enumerate}
        \end{enumerate}
    \end{enumerate}
    \subsection{潮汐能的影响因素}
    \begin{enumerate}
        \item 天文(天体引力)
        \item 地形
        \item 盛行风
        \item 径流(河流径流量大,入海口逆潮顶托抬高潮位)
        \item 地转偏向力
    \end{enumerate}
    \subsection{石油危机应对措施}
    \begin{enumerate}
        \item 拓宽石油进口渠道
        \item 加强战略石油储备
        \item 加强资源勘探,增加储量
        \item 开发寻找新能源(如:太阳能,风能,潮汐能)
        \item 节约利用,提高资源的利用效率
        \item 推广绿色出行方式,加强公共交通建设
    \end{enumerate}
    \subsection{煤炭开采,运输,利用中的环境问题与保护措施}
    \begin{enumerate}
        \item 采矿过程中的环境问题
        \begin{enumerate}
            \item 地下开采
            \begin{enumerate}
                \item 容易导致地表沉陷,地上建筑物毁损
                \item 措施
                \begin{enumerate}
                    \item 矿井回填(用碎石,沙,矸石等)
                \end{enumerate}
            \end{enumerate}
            \item 露天开采
            \begin{enumerate}
                \item 容易导致地表植被破坏
                \item 废料堆放占用大量土地,造成土地退化
                \item 造成水污染
                \item 措施
                \begin{enumerate}
                    \item 平整土地,覆土复垦,恢复矿区植被,保护水资源
                \end{enumerate}
            \end{enumerate}
        \end{enumerate}
        \item 运输中的环境问题
        \begin{enumerate}
            \item 增加运输压力
            \item 粉尘污染
            \item 措施
            \begin{enumerate}
                \item 集装箱,封闭运输,专用通道,坑口电站
            \end{enumerate}
        \item 利用中的环境问题
            \item 大气污染(固体悬浮颗粒,全球变暖,酸雨)
            \item 热污染,废弃物污染
            \item 措施
            \begin{enumerate}
                \item 推广洁净燃烧技术,提高煤炭利用效率
                \item 开发利用新能源,废弃物综合利用
            \end{enumerate}
        \end{enumerate}
    \end{enumerate}
    \subsection{水资源短缺的原因和措施}
    \begin{enumerate}
        \item 自然原因
        \begin{enumerate}
            \item 水资源总量有限
            \item 水资源分布时空不均
            \begin{enumerate}
                \item (时间上,降水季节分配不均,年际变化大)
                \item (空间上,南多北少,东多西少)
            \end{enumerate}
        \end{enumerate}
        \item 人为原因
        \begin{enumerate}
            \item 人口数量大,人均占有量少
            \item 水资源利用不当,水污染,水浪费严重,重复利用率低
            \item 节水意识淡薄,水价较低
            \item 植被,湿地破坏
        \end{enumerate}
        \item 水资源短缺措施
        \begin{enumerate}
            \item 开源
            \begin{enumerate}
                \item 合理开发和提取地下水
                \item 修建水库,跨流域调水
                \item 植树种草,涵养水源
                \item 海水淡化,人工增雨
            \end{enumerate}
            \item 节流
            \begin{enumerate}
                \item 节水农业,改进灌溉技术
                \item 提高水资源利用率,提高水价
                \item 防治水污染
                \item 提高节水意识
            \end{enumerate}
        \end{enumerate}
    \end{enumerate}
    \subsection{地形特征的描述}
        \begin{tabular}{|c|c|}
            \hline
            命题方向    &   内容    \\
            \hline
            地形类型    &   平原,高原,山地,丘陵,盆地    \\
            \hline
            地势    &    \makecell[c]{地势xx高xx低,地势自xx向xx倾斜\\地形崎岖(平坦)或地面起伏大(小)}   \\
            \hline
            海岸线    &    \makecell[c]{海岸线平直\\海岸线曲折,多半岛,岛屿等}   \\
            \hline
            特殊地貌    &    喀斯特地貌发育,冰川地貌发育   \\
            \hline
        \end{tabular}
    \subsection{地形对区域气温的影响}
    \begin{enumerate}
        \item 地势
        \begin{enumerate}
            \item 海拔越高,气温越低
            \item 高大山脉的水热会呈垂直差异分布,造成气候呈垂直差异分异
        \end{enumerate}
        \item 地形类型
        \begin{enumerate}
            \item 河谷地形热量不利于扩散,且盛行下沉气流,气温高于同纬度其他地区气温
            \item 盆地四周气温低,中间气温高
        \end{enumerate}
        \item 坡向
        \begin{enumerate}
            \item 冷空气迎风侧,冷空气堆积,气温低(逆温)
            \item 背风侧,山脉阻挡冷空气与焚风效应,气温高
            \item 山脉也可以阻挡沙漠的干热空气
        \end{enumerate}
        \item 其他
        \begin{enumerate}
            \item 同一海拔阳坡高于阴坡
            \item 同一海拔迎风坡低于背风坡
            \item 高山的顶部,由于海拔高,地形坡度大,阳坡,阴坡的水平距离缩小,热量交换越容易,气温差异越来越小
            \item 山顶因为陆地面积小,气温日较差小,高原日较差大
        \end{enumerate}
    \end{enumerate}
    \subsection{地形对区域蒸发的影响}
    \begin{enumerate}
        \item 地势
        \begin{enumerate}
            \item 海拔越高,气温越低,蒸发量越小
        \end{enumerate}
        \item 地形类型
        \begin{enumerate}
            \item 一般情况下,地形越平坦或者峡谷地形走向与风向一致时,越有利于蒸发
        \end{enumerate}
        \item 坡向
        \begin{enumerate}
            \item 阳坡蒸发量高于阴坡
        \end{enumerate}
    \end{enumerate}
    \subsection{地形对降水的影响}
    \begin{enumerate}
        \item 地势
        \begin{enumerate}
            \item 地势高大的山脉高原,会阻挡削弱盛行风的深入,影响降水
            \item 就同一地区不同山体而言,海拔更高的山体降水量较多
        \end{enumerate}
        \item 坡向
        \begin{enumerate}
            \item 山体的迎风坡大于背风坡
            \item 地形雨的形成对山体高度有一定的要求,一般要求相对高度高于500米
            \item 来自海洋盛行风的迎风坡,降水多
            \item 随着海拔高度的增加,降水量由少到多再到少
        \end{enumerate}
        \item 其他
        \begin{enumerate}
            \item 来自陆地的盛行风的迎风坡,降水少
            \item 无论来自海洋还是陆地的风,背风坡降水少,(随着高度的降低,降水量由多到少)
            \item 地形类型
            \begin{enumerate}
                \item 在盆地或谷地多形成夜雨
            \end{enumerate}
        \end{enumerate}
    \end{enumerate}
    \subsection{地形对风的影响}
    \begin{enumerate}
        \item 山脉走向影响风速,风向
        \begin{enumerate}
            \item 山脉与风向垂直,可降低风速
            \begin{enumerate}
                \item 北美南北走向的落基山脉阻挡西风的深入
                \item 中国东西走向的阴山,秦岭,南岭等对冬季风的阻挡明显
            \end{enumerate}
            \item 与风向平行,有利于风的深入
            \begin{enumerate}
                \item 北美中部大平原贯穿南北利于冷空气的南下和暖空气的北上
                \item 东西走向的阿尔卑斯山脉利于西风的深入
                \item 两座山脉之间或河谷之间可能形成狭管效应
            \end{enumerate}
        \end{enumerate}
        \item 地形类型影响风速,风向
        \begin{enumerate}
            \item 高大的山地高原对盛行风阻挡削弱作用明显
            \item 某区域大尺度的风向,在小区域如果遇到地形的阻挡,可能改变原来的风向
            \item 平原地区摩擦力小,风力大,且利于深入内陆
            \item 山谷风
        \end{enumerate}
    \end{enumerate}
    \subsection{地形对河流水系的影响}
    \begin{enumerate}
        \item 影响流向
        \begin{enumerate}
            \item 河流流向取决于地势高低,由地势高处流向地势低处
        \end{enumerate}
        \item 影响流域面积和水系形状
        \begin{enumerate}
            \item (山脉)分水岭决定流域面积
            \item 山地多放射状水系,盆地多为向心状水系,平原多树枝状水系
            \item 一般平原地区河网较密
        \end{enumerate}
        \item 影响河道剖面
        \begin{enumerate}
            \item 一般而言,山区(坡度大,落差大)河流多呈"V"字型
            \item 平原区(坡度小,落差小)河流多呈"U"字型
        \end{enumerate}
    \end{enumerate}
    \subsection{地形对水文的影响}
    \begin{enumerate}
        \item 影响流量,含沙量
        \begin{enumerate}
            \item 坡度大小影响下渗量及汇水速度,一定程度上影响地表和地下径流量
            \item (从地形角度看)山区坡度大,流速快,流水侵蚀,含沙量增大
            \item 平原区地形平坦,流速慢,泥沙沉积,含沙量小
        \end{enumerate}
        \item 影响流速,水能,航运
        \begin{enumerate}
            \item 一般而言,山区(坡度大,落差大)河流流速较快,水能较丰富,航运条件较差
            \item 平原区(坡度小,落差小)河流流速较慢,水能较缺乏,航运条件较好
        \end{enumerate}
    \end{enumerate}
    \subsection{地形对农业生产的影响}
    \begin{enumerate}
        \item 影响农业类型
        \begin{enumerate}
            \item 平原:种植业
            \item 高原:畜牧业
            \item 山区:林牧业
        \end{enumerate}
        \item 影响机械化水平和生产规模
        \item 影响生长种类,熟制,产量,品质
        \item 海拔,地势:热量,光照,昼夜温差
        \item 坡度:排水,灌溉,洪涝,土层深厚,土壤肥力
        \item 坡向:光照,热量,降水
    \end{enumerate}
    \subsection{交通线路的选线原则(即交通修建的地理意义)}
    \begin{enumerate}
        \item 自然因素
        \begin{enumerate}
            \item 地形
            \begin{enumerate}
                \item 平原
                \begin{enumerate}
                    \item 平原区地形对线路的限制较少,选线时要尽量少占耕地,处理好与农田水利建设,城镇发展的关系
                \end{enumerate}
                \item 山地
                \begin{enumerate}
                    \item 线路尽量沿等高线修筑,尽量避开地形复杂的地区,在陡坡上修成"之"字型弯曲或开凿隧道
                \end{enumerate}
            \end{enumerate}
            \item 水文
            \begin{enumerate}
                \item 线路应避开沼泽地,尽量避免跨越河流,以减少桥涵总长度
            \end{enumerate}
            \item 地质
            \begin{enumerate}
                \item 注意避开断层地带和滑坡,泥石流多发地区,特别是开凿隧道时尽量避开断层带,从背斜部位穿越
            \end{enumerate}
            \item 气候
            \begin{enumerate}
                \item 公路,铁路
                \begin{enumerate}
                    \item 防暴雨,洪涝,冻土,泥石流
                \end{enumerate}
                \item 水运,航空
                \begin{enumerate}
                    \item 防大雾,大风
                \end{enumerate}
            \end{enumerate}
        \end{enumerate}
        \item 社会经济因素
        \begin{enumerate}
            \item 人口
            \begin{enumerate}
                \item 联系较多的居民点,带动沿线经济的发展(适用于地方性道路)
                \item 穿越居民点较少,人口搬迁少
            \end{enumerate}
            \item 里程
            \begin{enumerate}
                \item 距离短,工程量小,节省运营时间,完善交通网
            \end{enumerate}
            \item 经济
            \begin{enumerate}
                \item 把资源优势变成经济优势,带动区域经济发展
            \end{enumerate}
            \item 带动相关产业的发展(促进就业,缓解就业压力)
            \item 加强沿线地区人员,物质,文化交流
            \item 带动少数民族脱贫致富,加强民族团结
            \item 巩固国防,利于社会长治久安
        \end{enumerate}
    \end{enumerate}
    \subsection{全球气候变暖的原因及措施}
    \begin{enumerate}
        \item 原因
        \begin{enumerate}
            \item 人为原因
            \begin{enumerate}
                \item 二氧化碳含量与日俱增,吸收地面长波辐射
                \begin{enumerate}
                    \item 工厂,交通工具,家庭炉灶大量燃烧煤,石油,天然气,排放大量二氧化碳
                    \item 森林被大量砍伐,森林吸收二氧化碳的能力削弱
                    \item 排入大气的氟氯烃,严重破坏臭氧层
                \end{enumerate}
                \item 射向地面紫外线增多
                \begin{enumerate}
                    \item 排入大气的氟氯烃,严重破坏臭氧层
                \end{enumerate}
            \end{enumerate}
            \item 自然原因
            \begin{enumerate}
                \item 目前地球正处于温暖期
            \end{enumerate}
        \end{enumerate}
        \item 措施
        \begin{enumerate}
            \item 增加温室气体吸收
            \begin{enumerate}
                \item 植树造林
                \item 减少森林植被破坏
            \end{enumerate}
            \item 适应气候变化
            \begin{enumerate}
                \item 培育农作物新品种
                \item 调整农业生产结构
                \item 建防护坝
            \end{enumerate}
            \item 技术手段
            \begin{enumerate}
                \item 节能技术
                \item 生物技术
                \item 固碳技术
            \end{enumerate}
            \item 控制温室气体排放
            \begin{enumerate}
                \item 改变能源结构
                \item 加快调整产业结构
                \item 提高能源利用率
            \end{enumerate}
            \item 政策手段
            \begin{enumerate}
                \item 公众参与
                \item 经济手段
                \item 直接控制
            \end{enumerate}
        \end{enumerate}
    \end{enumerate}
    \subsection{全球气候变暖对沿海地区的影响}
    \begin{enumerate}
        \item 影响
        \begin{enumerate}
            \item 海平面上升
            \begin{enumerate}
                \item 淹没沿海低地
                \item 地下水位升高,土壤盐碱化
                \item 海岸侵蚀加剧
                \item 影响沿海水产养殖业
                \item 破坏港口设备,影响航运
                \item 全球变暖
            \end{enumerate}
            \item 物种遭受损失
            \begin{enumerate}
                \item 自然生态系统不能适应变化了的环境
                \item 人类占用的土地限制了生态系统的自然迁移
            \end{enumerate}
            \item 海水温度变化及某些洋流的潜在变化->鱼类聚集地变化->某些渔场消失,某些渔场扩大
            \item 土地荒漠化->生态环境恶化
        \end{enumerate}
    \end{enumerate}
    \subsection{新能源优缺点对比}
        \begin{tabular}{|l|l|l}
            \hline
            类型 & 优点 & 缺点 \\
            \hline
            风能 & \makecell[l]{可再生,清洁无污染 \\ 蕴藏量大,分布广泛} & \makecell[l]{密度低(分散),不稳定 \\ 投资大 \\ 风力丰富地区与能源消费区不匹配} \\
            \hline
            太阳能 & 能量巨大,无污染,可再生 & \makecell[l]{能量比较分散 \\ 投资大 \\ 效率低,占地广,储能难 \\ 受天气,季节影响大不稳定} \\
            \hline
            地热能 & 可再生,清洁无污染 & 投资大,受地域限制 \\
            \hline
            生物能 & \makecell[l]{可再生,低污染 \\ 分布广泛,总量丰富} & 产量小,利用率低 \\
            \hline
            核能 & \makecell[l]{能量集中,巨大,地区适应性强 \\ 运转费用低,收益大} & \makecell[l]{非可再生资源,投资大 \\ 技术要求高,建设周期长} \\
            \hline
        \end{tabular}
    \subsection{新能源开发对区域发展的影响}
    \begin{enumerate}
        \item 有利影响
        \begin{enumerate}
            \item 缓解能源资源短缺的状况
            \item 改善能源消费结构
            \item 减轻大气污染状况,实现可持续发展
        \end{enumerate}
        \item 不利影响
        \begin{enumerate}
            \item 新能源往往开发难度较大,且不太稳定,易受到自然条件的影响
            \item 生物能源的开发抢占了耕地,影响区域粮食安全
        \end{enumerate}
    \end{enumerate}
    \subsection{盐场的区位条件}
    \begin{enumerate}
        \item 自然条件
        \begin{enumerate}
            \item 气候
            \begin{enumerate}
                \item 气温,降水,空气湿度,风力,日照时数,强度(天气)
            \end{enumerate}
            \item 地形
            \begin{enumerate}
                \item 地形平坦开阔,淤泥质海滩
            \end{enumerate}
            \item 海水盐度
        \end{enumerate}
        \item 社会经济条件
        \begin{enumerate}
            \item 市场,交通,劳动力,晒盐历史
        \end{enumerate}
    \end{enumerate}
    \subsection{台风致灾程度(危害程度)的因素}
    \begin{enumerate}
        \item 台风强度(如风力大小,降水强度等)
        \begin{enumerate}
            \item 下垫面状况(孕灾环境)
            \begin{enumerate}
                \item 如地形,通常平原地势低平,排水不畅,而且摩擦力小,风力强
                \item 山区地势起伏大,对风力的削弱作用强,但暴雨集中,易引起泥石流,山洪暴发等并发灾害
            \end{enumerate}
            \item 受灾区域的社会经济发展程度
            \begin{enumerate}
                \item 如经济发达,人口和城市密集则损失大
            \end{enumerate}
            \item 自然灾害预警系统的完善程度
            \begin{enumerate}
                \item 如灾害预报,应急撤退,救援的反应力等
            \end{enumerate}
            \item 人们的防灾意识
            \begin{enumerate}
                \item 如国土规划,城市建设,防灾工程等渗入的抗灾准备
            \end{enumerate}
        \end{enumerate}
    \end{enumerate}
    \subsection{洪涝灾害的成因}
    \begin{enumerate}
        \item 自然原因
        \begin{enumerate}
            \item 水系(支流的多少,干支流构成的形状,河道的弯曲度)
            \begin{enumerate}
                \item 缺少天然的入海河道(如淮河)
                \item 水系支流多(扇形水系,主支流同时入汛,加剧洪涝)
                \item 河道弯曲
            \end{enumerate}
            \item 水文(汛期长短,流量大小及变化,含沙量大小及河床泥沙淤积情况,有无凌汛现象)
            \item 气候(降水量的大小及变率,台风等)
            \item 地形(地势平坦程度,水流不畅)
            \item 其他因素(海水顶托等)
        \end{enumerate}
        \item 人为原因
        \begin{enumerate}
            \item 植被破坏
            \item 围湖造田
            \item 工程建设
        \end{enumerate}
    \end{enumerate}
    \subsection{洪涝灾害的防治措施}
    \begin{enumerate}
        \item 工程措施
        \begin{enumerate}
            \item 上游
            \begin{enumerate}
                \item 植树造林,修建水库
            \end{enumerate}
            \item 中游
            \begin{enumerate}
                \item 修建水库和分洪,蓄洪工程,裁弯取直疏浚河道,退耕还湖
            \end{enumerate}
            \item 下游
            \item 加固河堤,疏浚河道,退耕还湖,开挖入海河道
        \end{enumerate}
        \item 非工程措施
        \begin{enumerate}
            \item 加强防灾减灾意识
            \item 加强洪水的监测和预报
            \item 制定防洪防灾预警措施
            \item 加强防洪防灾的法治建设
        \end{enumerate}
    \end{enumerate}
    \subsection{东北华北春旱的成因}
    \begin{enumerate}
        \item 自然原因
        \begin{enumerate}
            \item 春季气温回升快,大风日数多,蒸发旺盛,而雨季未到,降水稀少
        \end{enumerate}
        \item 人类活动
        \begin{enumerate}
            \item 春季正值东北农作物播种期和华北冬小麦生长的关键期
        \end{enumerate}
    \end{enumerate}
    \subsection{长江流域伏旱的成因}
    \begin{enumerate}
        \item 自然原因
        \begin{enumerate}
            \item 受副高控制,盛行下沉气流,降水稀少
            \item 日照强烈,蒸发旺盛
        \end{enumerate}
        \item 人类活动
        \begin{enumerate}
            \item 作物生长期,水电和城市用水需求量大
        \end{enumerate}
    \end{enumerate}
    \subsection{西南地区冬春连旱的成因}
    \begin{enumerate}
        \item 自然原因
        \begin{enumerate}
            \item 冬春季降雨少
            \item 气温较高,蒸发旺盛
            \item 地形崎岖,地表起伏大,地表水存留时间短
            \item 多为喀斯特地貌,多地下暗河,地表水储藏条件差
            \item 土层薄,水源涵养能力差
        \end{enumerate}
        \item 人类活动
        \begin{enumerate}
            \item 水利设施缺乏,老化和损坏严重
            \item 人们习惯靠天吃水,缺乏水资源的节约和保护意识
        \end{enumerate}
    \end{enumerate}
    \subsection{干旱的危害和防治措施}
    \begin{enumerate}
        \item 开源
        \begin{enumerate}
            \item 合理开发和提取地下水
            \item 修建水库,跨流域调水
            \item 植树种草,涵养水源
            \item 海水淡化
            \item 人工增雨
        \end{enumerate}
        \item 节流
        \begin{enumerate}
            \item 节水农业,改进灌溉技术
            \item 提高水资源利用率,提高水价
            \item 防治水污染
            \item 提高节水意识
        \end{enumerate}
    \end{enumerate}
    \subsection{寒潮的防御措施}
    \begin{enumerate}
        \item 发布准确的信息和警报
        \item 做好防寒准备
        \begin{enumerate}
            \item 农作物提前抢收
            \item 人造烟幕,田地浇水
            \item 搭建大棚,覆盖地膜
            \item 加固圈舍,为牲畜提前准备饲料等
        \end{enumerate}
        \item 推广和培育耐寒品种
        \item 海上船只及时回港
    \end{enumerate}
    \subsection{寒潮的影响}
    \begin{enumerate}
        \item 弊
        \begin{enumerate}
            \item 对农作物造成冻害(强烈降温)
            \item 吹翻船只,摧毁建筑物,破坏农场(大风)
            \item 压断电线,折断电线杆(大雪,冻雨)
            \item 寒潮带来的雨雪和冰冻天气对交通运输危害不小
            \item 寒潮袭来对人体健康危害很大
        \end{enumerate}
        \item 利
        \begin{enumerate}
            \item 寒潮有助于地球表面热量交换
            \item 寒潮带来的低温,可大量杀死潜伏在土中过冬的害虫和病菌,或抑制其滋生,减轻来年的病虫害
            \item 雨雪天气,缓解了冬天的旱情,使农作物受益
            \item 大雪覆盖在越冬农作物上,就像棉被一样起到抗寒保温作用
            \item 有助于自然界的生态保持平衡,保持物种的繁茂
        \end{enumerate}
    \end{enumerate}
    \subsection{沙尘暴的成因}
    \begin{enumerate}
        \item 自然原因
        \begin{enumerate}
            \item 气候干旱,降水少
            \item 地表植被稀少
            \item 多松散碎屑物
            \item 快行冷锋天气影响,春季大风日数多,风力的吹扬
        \end{enumerate}
        \item 人为原因
        \begin{enumerate}
            \item 过度放牧,过度樵采,过度开垦,破坏水源,工矿交通建设等
        \end{enumerate}
        \item 治理措施
        \begin{enumerate}
            \item 制定草场保护的法律,法规,加强管理
            \item 控制载畜量
            \item 营造"三北防护林"
            \item 退耕还林,还牧
            \item 建设人工草场,推广轮牧等
        \end{enumerate}
    \end{enumerate}
    \subsection{火山喷发对地理环境的影响}
    \begin{enumerate}
        \item 不利影响
        \begin{enumerate}
            \item 改变气候
            \begin{enumerate}
                \item 火山喷发出大量火山灰颗粒,遮挡太阳辐射,导致地表气温下降,使农作物减产
            \end{enumerate}
            \item 破坏环境
            \begin{enumerate}
                \item 火山喷发出大量含有硫磺的有毒气体,形成酸雨,火山灰会阻碍动植物呼吸,造成动植物死亡
            \end{enumerate}
            \item 影响生活
            \begin{enumerate}
                \item 喷发的火山灰与暴雨结合形成泥石流能冲毁道路,桥梁,影响交通,淹没农田城镇,造成人员财产损失
            \end{enumerate}
        \end{enumerate}
        \item 有利影响
        \begin{enumerate}
            \item 形成丰富的矿产资源
            \item 地热资源
            \item 火山灰富有矿物质,使土壤肥沃
            \item 利用火山景观和温泉,发展旅游业,岩浆岩可作为建筑材料
        \end{enumerate}
    \end{enumerate}
    \subsection{地震烈度的影响因素}
    \begin{enumerate}
        \item 自然
        \begin{enumerate}
            \item 震级大小
            \item 震源深浅
            \item 震中距距离
            \item 地形,地质状况
        \end{enumerate}
        \item 人文
        \begin{enumerate}
            \item 建筑物抗震度
            \item 人口密度
            \item 经济发展程度
            \item 地震发生时间
            \item 防灾减灾意识
        \end{enumerate}
    \end{enumerate}
    \subsection{滑坡的成因和防御措施}
    \begin{enumerate}
        \item 成因
        \begin{enumerate}
            \item 地形
            \begin{enumerate}
                \item 斜坡,地形起伏大气候
                \item 降水强度大,久旱遇暴雨或融雪融冰
            \end{enumerate}
            \item 地质
            \begin{enumerate}
                \item 断裂带岩石破碎
                \item 地质结构不稳定
            \end{enumerate}
            \item 地表
            \begin{enumerate}
                \item 植被覆盖率低
            \end{enumerate}
            \item 人为原因
            \begin{enumerate}
                \item 开垦,开矿,工程建设
            \end{enumerate}
        \end{enumerate}
        \item 措施
        \begin{enumerate}
            \item 退耕还林,还草,恢复生态环境
            \item 排水防渗
            \item 构筑护坡工程(支挡抗滑桩,抗滑墙,锚固山体)
        \end{enumerate}
    \end{enumerate}
    \subsection{泥石流的成因和防御措施}
    \begin{enumerate}
        \item 成因
        \begin{enumerate}
            \item 地形
            地形崎岖,坡陡谷深的山区
            \item 大量的松散碎屑物质
            \begin{enumerate}
                \item 构造破碎(地震,断裂带),风化物等
            \end{enumerate}
            \item  洪水
            \begin{enumerate}
                \item 暴雨或冰雪融水汇集
            \end{enumerate}
            \item 植被
            \begin{enumerate}
                \item 覆盖率低,山坡表层缺少保护
            \end{enumerate}
            \item 人类活动
            \begin{enumerate}
                \item 开垦,开矿等会诱发或加剧泥石流
            \end{enumerate}
        \end{enumerate}
        \item 措施
        \begin{enumerate}
            \item 退耕还林还草,植树种草,恢复生态
            \item 修筑拦水坝
            \item 修筑排导槽,疏导泥石流物质到特定位置等
            \item 构筑护坡工程
        \end{enumerate}
    \end{enumerate}
    \subsection{地面沉降的成因和防御措施}
    \begin{enumerate}
        \item 成因
        \begin{enumerate}
            \item 自然原因
            \begin{enumerate}
                \item 构造升降运动,地震,火山活动,松软地基,地面加载
            \end{enumerate}
            \item 人为原因
            \begin{enumerate}
                \item 开采地下水,油气资源,地面加载(如城市建设)
            \end{enumerate}
        \end{enumerate}
        \item 措施
        \begin{enumerate}
            \item 严格控制地下水开采规模
            \item 雨季(汛期)回补地下水
            \item 节约用水
            \item 兴修水利,跨流域调水,以地表水代替地下水资源
            \item 油气田通过人工回灌,对抽汲的液体进行等体积替换等
        \end{enumerate}
    \end{enumerate}
    \subsection{崩塌的成因和防御措施}
    \begin{enumerate}
        \item 成因
        \begin{enumerate}
            \item 不稳定的地质构造
            \item 陡峭的地形地势
            \item 岩石风化强烈(物理,化学,生物风化)
            \item 地震,河流侵蚀坡脚,工程建设,开矿等易诱发崩塌
        \end{enumerate}
        \item 措施
        \begin{enumerate}
            \item 构筑护坡工程(锚固,支撑)
            \item 危岩裂缝填充,灌浆加固
            \item 修筑排导槽,排出岩体水分
            \item 拦石墙,拦石栅栏及森林防护
        \end{enumerate}
    \end{enumerate}
    \subsection{蝗灾的防御措施}
    \begin{enumerate}
        \item 加强蝗情监测,及时掌握,反映蝗虫发生和防治动态
        \item 农药灭虫
        \item 投放天敌
        \item 人工诱捕
        \item 点火焚烧
    \end{enumerate}
    \subsection{森林火灾的成因}
    \begin{enumerate}
        \item 火源
            \item 人为火源,自然火源
        \item 可燃物
            \item 枯枝落叶,乔木,灌木等
        \item 火险天气
            \item 高温,干燥(降水少,蒸发旺盛),大风
    \end{enumerate}
    \subsection{影响人口自然增长的因素}
    \begin{enumerate}
        \item 自然因素
        \begin{enumerate}
            \item 生物学规律的制约(育龄妇女)
            \item 自然环境和自然灾害(通过死亡率来影响)
        \end{enumerate}
        \item 经济基础
        \begin{enumerate}
            \item 经济发达程度
            \begin{enumerate}
                \item 工业化程度较低的地区,出生率高
            \end{enumerate}
            \item 文化教育水平
            \begin{enumerate}
                \item 女性人口受教育水平越高,出生率越低
            \end{enumerate}
            \item 医疗卫生条件
        \end{enumerate}
        \item 上层建筑
        \begin{enumerate}
            \item 婚姻生育观
            \item 宗教信仰
            \item 风俗习惯
            \item 人口政策
        \end{enumerate}
        \item 归根结底取决于生产力的发展水平
    \end{enumerate}
    \subsection{人口增长过快带来的问题和相应措施}
    \begin{enumerate}
        \item 问题
        \begin{enumerate}
            \item 粮食供给不足
            \item 就业问题严重
            \item 人民生活贫困(经济)
            \item 环境和资源压力大
        \end{enumerate}
        \item 措施
        \begin{enumerate}
            \item 实行计划生育政策,控制人口数量,提高人口素质
        \end{enumerate}
    \end{enumerate}
    \subsection{人口增长过慢带来的问题和相应措施}
    \begin{enumerate}
        \item 问题
        \begin{enumerate}
            \item 人口老龄化
            \item 社会经济负担加重
            \item 劳动力不足,国防兵源不足
            \item 老年人生活困难
        \end{enumerate}
        \item 措施
        \begin{enumerate}
            \item 鼓励生育,接纳海外移民,延迟退休,完善社会养老保障体系
        \end{enumerate}
    \end{enumerate}
    \subsection{我国人口问题及措施}
    \begin{table}[h]
        \begin{tabular}{|p{35mm}|l|l|}
            \hline
            \multicolumn{2}{c}{我国人口问题及措施} \\
            \hline
            人口性别结构不合理 & 实行计划生育政策,转变人口生育观念 \\
            \hline
            人口素质较低 & 大力发展科技教育,提高人口素质 \\
            \hline
            人口地区分布不平衡 & 加强经济建设,提高和改善落后地区的社会经济条件 \\
            \hline
            人口流动规模大 & 大力发展交通和经济,缩小地区发展差距,增加当地就业机会 \\
            \hline
            空巢老人,留守儿童,独生子女家庭增多 & 发展社会保障事业,增加就地就业岗位和机会等 \\
            \hline
        \end{tabular}
    \end{table}
    \subsection{人口性别结构不合理}
    \begin{enumerate}
        \item 衡量(性别比)
        \begin{enumerate}
            \item 即用100位女性对应的男性数来衡量,反映该地区或国家人口的性别结构是否合理或协调
        \end{enumerate}
        \item 问题
        \begin{enumerate}
            \item 性别比失衡会造成婚姻,家庭和社会的不稳定,不利于社会经济的发展
        \end{enumerate}
        \item 措施
        \begin{enumerate}
            \item 加强对非法胎儿性别鉴别的监管
            \item 大力发展生产力,提高妇女地位
            \item 完善养老保障体系
            \item 加强宣传教育,转变婚育观念
        \end{enumerate}
    \end{enumerate}
    \subsection{环境人口容量的影响因素}
    \begin{enumerate}
        \item 资源丰富程度
        \item 科技发展水平
        \item 经济发达程度
        \item 地区开放程度
        \item 生活消费水平
        \item 人口的受教育水平
    \end{enumerate}
    \subsection{人口迁移的影响因素}
    \begin{enumerate}
        \item 自然生态环境因素
        \begin{enumerate}
            \item 气候,土壤,水和矿产资源,自然灾害等
        \end{enumerate}
        \item 经济因素
        \begin{enumerate}
            \item 经济发展水平,就业,收入,生活条件,交通等(如我国涌向东部沿海的民工潮)
        \end{enumerate}
        \item 政治因素
        \begin{enumerate}
            \item 国家政策,战争等
        \end{enumerate}
        \item 社会文化因素
        \begin{enumerate}
            \item 文化教育,家庭婚姻,宗教信仰,个人动机等
        \end{enumerate}
    \end{enumerate}
    \subsection{人口迁移的影响}
        \begin{tabular}{|c|c|}
            \hline
            迁出地-利 & 迁出地-弊 \\
            \hline
            \makecell[l]{加强与外界的联系与交流\\ 缓解人地矛盾\\增加就业机会\\增加经济收入} &  人才流失缺乏劳动力 \\
            \hline
            迁入地-利 & 迁入地-弊 \\
            \hline
            提供了大量劳动力,促进经济的发展,城市化 &   增加了城市负担,对环境,治安等有影响 \\
            \hline
        \end{tabular}
    \subsection{城市区位因素}
    \begin{enumerate}
        \item 地形
        \begin{enumerate}
            \item 平原
            \item 高原(热带)
            \item 山区(河谷)
        \end{enumerate}
        \item 气候
        \begin{enumerate}
            \item 气温(中低纬度)
            \item 降水(中低纬度)
        \end{enumerate}
        \item 河流
        \begin{enumerate}
            \item 供水
            \item 运输
            \item 防卫
        \end{enumerate}
        \item 自然资源
        \item 交通
        \item 政治
        \item 军事
        \item 宗教
        \item 旅游
        \item 科技
    \end{enumerate}
    \subsection{城市功能分区的影响因素}
    \begin{enumerate}
        \item 经济因素
        \begin{enumerate}
            \item 距离市中心的远近,交通的通达性
        \end{enumerate}
        \item 历史原因
        \begin{enumerate}
            \item 原有基础很大程度上决定了现在的功能区现状
        \end{enumerate}
        \item 自然因素
        \begin{enumerate}
            \item 土地的面积与形状,地形起伏程度与坡度大小,水文和气象条件等
        \end{enumerate}
        \item 社会原因
        \begin{enumerate}
            \item 收入差异,知名度,种族和宗教
        \end{enumerate}
        \item 行政因素
        \begin{enumerate}
            \item 政府引导或划分不同的功能区
        \end{enumerate}
    \end{enumerate}
    \subsection{中心商务区的特征}
    \begin{enumerate}
        \item 经济活动最繁忙
        \item 人口数量昼夜差别大
        \item 建筑物高大稠密
        \item 内部存在明显分区
    \end{enumerate}
    \subsection{郊区城市化与逆城市化的区别,逆城市化的原因}
    \begin{enumerate}
        \item 区别
        \begin{enumerate}
            \item 郊区城市化
            \begin{enumerate}
                \item 中心城区和乡村小镇的人和产业向城市郊区转移,这是城市化加速过程中出现的现象
            \end{enumerate}
            \item 逆城市化
            \begin{enumerate}
                \item 中心城区和城市郊区的人向乡村,小城镇转移,这是城市化后期过程中出现的现象
            \end{enumerate}
        \end{enumerate}
        \item 逆城市化原因
        \begin{enumerate}
            \item 城市环境恶化,交通拥挤,住房紧张等
            \item 乡村环境优美,地价便宜,交通改善,基础设施完善
        \end{enumerate}
    \end{enumerate}
    \subsection{城市化过程对自然地理环境的影响}
    \begin{enumerate}
        \item 地形
        \begin{enumerate}
            \item 对原来的地形地貌进行改造,趋于平坦
            \item 问题:容易造成水土流失,滑坡,泥石流等地质灾害
        \end{enumerate}
        \item 气候
        \begin{enumerate}
            \item 气温:热岛效应
            \item 降水:雨岛效应
            \item 问题:城市风,扩大污染
        \end{enumerate}
        \item 水文
        \begin{enumerate}
            \item 破环原有的河网系统,原有河流被填满或分割城成断头河或死水河,水系紊乱,地下水减少
            \item 问题:暴雨时排水不畅,造成地面积水,残留河道容易因富营养化而发黑变臭
        \end{enumerate}
        \item 生态
        \begin{enumerate}
            \item 城市生产生活污染,尤其是工业三废,干扰和破坏环境生态
            \item 问题:脆弱的生态系统,城市是人类对自然环境影响和改变最大的地方
        \end{enumerate}
    \end{enumerate}
    \subsection{城市化问题的表现及措施}
    \begin{enumerate}
        \item 环境问题
        \begin{enumerate}
            \item 资源短缺
            \begin{enumerate}
                \item 耕地减少,水资源短缺
            \end{enumerate}
            \item 环境污染
            \begin{enumerate}
                \item 大气污染
                \item 水污染
                \item 噪声污染
                \item 固体废弃物污染
                \item 光化学污染
                \item 辐辐污染
            \end{enumerate}
            \item 生态破坏
            \begin{enumerate}
                \item 洪涝加剧
                \item 过度开采地下水,地下水位下降,地面沉降
                \item 热岛效应
                \item 生物多样性减少
                \item 城市风
            \end{enumerate}
            \item 社会问题
            \begin{enumerate}
                \item 城市交通问题
                \begin{enumerate}
                    \item 交通拥堵
                    \item 停车紧张
                \end{enumerate}
                \item 城市住宅问题
                \begin{enumerate}
                    \item 房价高,出现贫民窟,棚户区等
                \end{enumerate}
                \item 城市就业问题
                \begin{enumerate}
                    \item 就业困难失业现象严重
                    \item 失业者,技术过时者和缺乏充分教育的群体增加
                \end{enumerate}
                \item 社会治安间题
                \begin{enumerate}
                    \item 社会治安差,犯罪率高
                \end{enumerate}
            \end{enumerate}
        \end{enumerate}
    \end{enumerate}
    \subsection{卫星城的功能}
    \begin{enumerate}
        \item 分担城市职能
        \item 缓解城市土地,交通压力
        \item 有利于保护和改善城市环境
        \item 促进城市合理化发展
    \end{enumerate}
    \subsection{大城市的辐射功能}
    \begin{enumerate}
        \item 吸引周边劳动力来大城市工作
        \item 促进周边城市的经济的发展
        \item 大城市的高新科技,管理经验等向外辐射,提高周边地区经济水平
        \item 文化等带动作用,如新兴文化风潮由大城市向周边地区扩散,传播等
        \item 部分产业转移到周边地区,推动大城市与周边小城市的产业分工与合作
        \item 大城市交通网络发达,与周边地区联系密切,带动中小企业发展,形成卫星城
    \end{enumerate}
    \subsection{产业活动中的地域联系}
    \begin{enumerate}
        \item 生产协作
        \begin{enumerate}
            \item 多道工序之间的联系
            \item 零部件工厂之间的联系
        \end{enumerate}
        \item 商贸联系
        \begin{enumerate}
            \item 区域贸易
            \item 国际贸易
        \end{enumerate}
        \item 科技与信息联系
        \begin{enumerate}
            \item 加强经济协作,加大技术交流
            \item 交流和共享信息资源,内部管理和对外联系网络化
        \end{enumerate}
    \end{enumerate}
    \subsection{物流过程中的商贸联系环节及物流业的作用}
    \begin{enumerate}
        \item 商贸联系环节
        \begin{enumerate}
            \item 采购,运输,仓储,包装,配送,用户
        \end{enumerate}
        \item 作用
        \begin{enumerate}
            \item 减少企业库存
            \item 降低运营成本
            \item 提高经济效益
        \end{enumerate}
    \end{enumerate}
    \subsection{科技与信息联系的作用}
    \begin{enumerate}
        \item 作用
        \begin{enumerate}
            \item 准确把握市场
            \item 及时获得技术创新的信息
        \end{enumerate}
    \end{enumerate}
    \subsection{农业区位}
    \begin{enumerate}
        \item 自然条件
        \begin{enumerate}
            \item 气候(光照,热量,降水,水热配合,温差,灾害等)
            \item 地形(类型,海拔,坡度,坡向)
            \item 水源(数量,水质,取水便捷程度)
            \item 土壤(肥瘦,厚度,水热,颗粒(透气性,保水性),酸碱等)
        \end{enumerate}
        \item 农业技术
        \begin{enumerate}
            \item 劳动力(数量,质量(经验,技术),价格)
            \item 技术装备(农机)
            \item 生产技术(育种,栽培(垄作,覆盖),水肥控制,病虫害防疗,区域专门化)
            \item 种植方式(单,间,套,混,连作)(光热水土肥虫草)
            \item 耕作制度,保鲜冷藏技术等
        \end{enumerate}
        \item 社会经济
        \begin{enumerate}
            \item 市场(规模,空间远近,时间(上市早迟,反季节),竞争力)
            \item 交通(方式,组合,通达度,价格)
            \item 政策(支持,限制,补贴)
            \item 土地(价格,规模,位置,开发难度)
            \item 历史经验
            \item 农业基础(发展基础,基础设施)
        \end{enumerate}
    \end{enumerate}
    \subsection{光照对农业的影响}
    \begin{enumerate}
        \item 时间长短和强度影响光合作用的强弱
        \item 光照强
        \begin{enumerate}
            \item 有利影响
            \begin{enumerate}
                \item 有利于农作物的光合作用,产生更多有机物
                \item 利于作物着色
            \end{enumerate}
            \item 不利影响
            \begin{enumerate}
                \item 作物容易灼伤
                \item 生长周期短,有机质积累少
            \end{enumerate}
        \end{enumerate}
        \item 光照弱
        \begin{enumerate}
            \item 有利影响
            \begin{enumerate}
                \item 生长周期长,品质好
            \end{enumerate}
            \item 不利影响
            \begin{enumerate}
                \item 光照弱,不利于谷物的生长
                \item 作物着色差,卖相差
            \end{enumerate}
        \end{enumerate}
        \item 光照时间长
        \begin{enumerate}
            \item 在一定程度上可弥补热量不足
        \end{enumerate}
    \end{enumerate}
    \subsection{热量,气温对农业的影响}
    \begin{enumerate}
        \item 作物种类,作物熟制,分布范围,品质产量,生长发育
        \item 气温高
        \begin{enumerate}
            \item 有利影响
            \begin{enumerate}
                \item 气病虫害少,施用农药少
                \item 农作物的生长周期长,积累的有机质多
                \item 有机质分解慢,土壤肥沃
                \item 积雪覆盖时间长,春季积雪融化,既缓解春旱,又可以改善土壤墒情
            \end{enumerate}
            \item 不利影响
            \begin{enumerate}
                \item 热量不足,生长期短,农作物只能一年一熟,甚至无法生长
            \end{enumerate}
            \item 温差
            \begin{enumerate}
                \item 气温日较差大,有利于农作物的营养物质积累,农作物品质好
                \item 气温日较差小,农作物的品质较差
            \end{enumerate}
        \end{enumerate}
    \end{enumerate}
    \subsection{降水对农业的影响}
    \begin{enumerate}
        \item 降水特征
        \begin{enumerate}
            \item 农业生产特点
            \begin{enumerate}
                \item 季节性,周期性,地域性
            \end{enumerate}
            \item 作物种类
            \begin{enumerate}
                \item 喜湿作物,耐旱作物
            \end{enumerate}
            \item 农业类型
            \begin{enumerate}
                \item 种植业,林业,渔业,畜牧业
            \end{enumerate}
            \item 耕地类型
            \begin{enumerate}
                \item 水田,旱地,灌概地
            \end{enumerate}
        \end{enumerate}
        \item 降水变化
        \begin{enumerate}
            \item 季节变化
            \begin{enumerate}
                \item 年雨型
                \begin{enumerate}
                    \item 有利:降水丰富,利于作物生长
                    \item 不利:雨天多,雨量大,光照不足
                \end{enumerate}
                \item 少雨型
                \begin{enumerate}
                    \item 有利:光照充足,昼夜温差大,利于有机质积累
                    \item 不利:降水少,水源不足
                \end{enumerate}
                \item 夏雨型
                \begin{enumerate}
                    \item 有利:雨热同期利于作物生长
                    \item 不利:旱涝灾害频发
                \end{enumerate}
                \item 冬雨型
                \begin{enumerate}
                    \item 雨热不同期,作物生长期降水少,对作物生长不利
                \end{enumerate}
            \end{enumerate}
            \item 年际变化
            \begin{enumerate}
                \item 农作物产量和品质
            \end{enumerate}
            \item 日变化
            \begin{enumerate}
                \item 昼晴夜雨
                \begin{enumerate}
                    \item 有利
                    \begin{itemize}
                        \item 昼晴:白天气温高,日照强,有机质积累多
                        \item 夜雨:夜间气温低,减少营养物质消耗
                        \item 夜间气温低,蒸发蒸腾弱,利于作物生长
                    \end{itemize}
                    \item 不利
                    \begin{itemize}
                        \item 降水过多,引发泥石流等地质灾害
                    \end{itemize}
                \end{enumerate}
            \end{enumerate}
        \end{enumerate}
    \end{enumerate}
    \subsection{地形对农业的影响}
    \begin{enumerate}
        \item 宏观地形
        \begin{enumerate}
            \item 平原
            \begin{enumerate}
                \item 种植业:农业类型单一,生产规模大
                \item 专业化,规模化,商品化
            \end{enumerate}
            \item 高原
            \begin{enumerate}
                \item 畜牧业,种值业(河谷,盆地地区)
            \end{enumerate}
            \item 山地,丘陵
            \begin{enumerate}
                \item 林果业,畜牧业,立体衣业(农业类型多样)
                \item 生产观懊和机械化小平相对较低(农业生产技术要求高)
            \end{enumerate}
        \end{enumerate}
        \item 微观地形
        \begin{enumerate}
            \item 海拔
            \begin{enumerate}
                \item 海拔较高的地区
                \begin{enumerate}
                    \item 有利
                    \begin{itemize}
                        \item 光照强,利于光合作用
                        \item 气温低,农作物生长期长,病虫害较少
                        \item 昼夜温差大,利于有机质的积累
                        \item 农作物生长慢,上市晚,错季上市,市场竞争力强
                    \end{itemize}
                    \item 不利
                    \begin{itemize}
                        \item 气温低,热量不足,复种指数和产量较低
                    \end{itemize}
                \end{enumerate}
            \end{enumerate}
            \item 山坡(坡向,坡度)
            \begin{enumerate}
                \item 坡向
                \begin{enumerate}
                    \item 阴阳坡
                    \begin{itemize}
                        \item 一般而言,同一海拔,山体阳坡光热条件好于阴坡
                        \item 阴湿度大于阳坡
                    \end{itemize}
                    \item 迎风/背风坡:降水,气温
                \end{enumerate}
                \item 坡度
                \begin{enumerate}
                    \item 开发难度:陡坡大于缓坡
                    \item 水分状况:陡坡排水好,不易发生洪涝,但水土保持差
                    \item 土壤状况:缓坡土层厚,肥力高
                    \item 发展方向:陡坡发展水土保持林
                    \item 缓坡修梯田或者林果业
                \end{enumerate}
            \end{enumerate}
            \item 河谷(河谷,阶地)
            \begin{enumerate}
                \item 河谷
                \begin{enumerate}
                    \item 青藏高原:湟水谷地
                    \begin{itemize}
                        \item 土壤肥沃,水源充足
                        \item 热量相对较好,地势平坦
                    \end{itemize}
                    \item 橫断山区:干热河谷
                    \begin{itemize}
                        \item 高温,低湿
                    \end{itemize}
                    \item (影响农作物种类,分布,质量)
                \end{enumerate}
                \item 阶地
                \begin{enumerate}
                    \item 地势平坦,泥沙沉积,土壤肥沃,临近河流,利于灌溉,
                    \item 相对河漫滩地势较高,洪水威胁小
                \end{enumerate}
            \end{enumerate}
        \end{enumerate}
    \end{enumerate}
    \subsection{土壤对农业的影响}
    \begin{enumerate}
        \item 肥力
        \begin{enumerate}
            \item 有机质,矿物质的含量高低
        \end{enumerate}
        \item 土温
        \begin{enumerate}
            \item 影植物的生长和呼吸
        \end{enumerate}
        \item 厚薄
        \begin{enumerate}
            \item 冲积士,土层深厚
        \end{enumerate}
        \item 颗粒结构
        \begin{enumerate}
            \item 透气性保水性
        \end{enumerate}
        \item 酸碱度
        \begin{enumerate}
            \item 影响矿物盐类的溶解度
            \item 微生物生活
            \item 植物的营养
        \end{enumerate}
    \end{enumerate}
    \subsection{水源对农业的影响}
    \begin{enumerate}
        \item 数量
        \item 水质
        \item 取水便利度
    \end{enumerate}
    \subsection{市场对农业的影响}
    \begin{enumerate}
        \item 市场是农业类型和规模的决定性因素
        \item 市场时空差异
        \begin{enumerate}
            \item 时间上
            \begin{enumerate}
                \item 上市时间(反季节上市或错峰上市)
            \end{enumerate}
            \item 空间上
            \begin{enumerate}
                \item 市场距离和交通便捷程度交通通达度
                \item 运输需求与运输成本
            \end{enumerate}
        \end{enumerate}
        \item 市场需求
        \begin{enumerate}
            \item 供求关系
            \item 销售价格
        \end{enumerate}
        \item 产品市场竞争力
        \begin{enumerate}
            \item 农作物品牌与质量
            \begin{enumerate}
                \item 品质好(绿色产品),市场需求量大,价格高
            \end{enumerate}
            \item 成本
            \begin{enumerate}
                \item 运输成本,生产成本(劳动力,低价)仓储成本,营销成本
            \end{enumerate}
            \item 上市时间
            \begin{enumerate}
                \item 上市时间早,上市晚或者上市时间长,价格高
            \end{enumerate}
            \item 产量与市场的供需关系
            \begin{enumerate}
                \item 供不应求,价格高
                \item 供过于求,价格低
            \end{enumerate}
            \item 错开上市时间,产品种类,售后服务等
        \end{enumerate}
    \end{enumerate}
    \subsection{交通对农业的影响}
    \begin{enumerate}
        \item 交通运输推动商品农业的发展,促进农业生产的区域化,专业化
        \item 交通条件改善,可缩短农产品运输时间,扩大农产品销售市场(类型,规模,结构)
        \item 交通便利
        \begin{enumerate}
            \item 便于农业生产资料的运进和农产品的运出
            \item 便于扩大市场范围,拓宽销路
            \item 节约运输成本
            \item 缩短运输时间,提高产品上市时间
            \item 冷藏保鲜运输,利于保证产品质量
            \item 延长供应时间,保证稳定供应
        \end{enumerate}
    \end{enumerate}
    \subsection{农业技术对农业的影响}
    \begin{enumerate}
        \item 劳动力(数量,质量(经验,技术),价格)
        \item 技术装备(农机)
        \item 生产技术(育种,栽培,水肥控制,病虫害防治,区域专门化)
        \item 种植方式(单,间,套,混,连作):光热水土肥虫草
        \item 耕作制度
    \end{enumerate}
    \subsection{间作套种的优点}
    \begin{enumerate}
        \item 有利于改变农作物单一的局面,实现农业产业结构的调整
        \item 增强抵御市场风险的能力,提高对市场的适应能力,确保农民收入的稳定
        \item 改良土壤,提高土壤的肥力(特别是豆科作物),增强应对自然实害的能力
        \item 提高土地的利用率,实现一地多用,提高了经济效益
        \item 增加植被覆盖率,减少水土流失
        \item 提高作物品质(具体依据材料分折)
        \item 形成有利于另一种作物生长的小气候
    \end{enumerate}
    \subsection{轮作的效益}
    \begin{enumerate}
        \item 经济
        \begin{enumerate}
            \item 节本:节省生产成本减少化肥,衣药等物质投入
            \item 增收:增加产量增加产值
        \end{enumerate}
        \item 生态
        \begin{enumerate}
            \item 节能:节约能源资源包括充分利用自然资源,节省能源资源投和提高能量转化利用效率等
            \item 促长
            \begin{enumerate}
                \item 促进作物生长发育为实现作物高产高打下坚实甚础
            \end{enumerate}
            \item 改土
            \begin{enumerate}
                \item 改菩士壤理,化性状和生物学性状培肥地力
            \end{enumerate}
            \item 减害
            \begin{enumerate}
                \item 减少病,虫,杂草危害,消除"毒"(即土壤中有毒物质的危害),从而可大大减少农药,除草剂等化学药品的使用量
                \item 不仅保护了农田生态环境,还有利于生产"健康食品"(生产无公害食品,绿色食品和有机食品)
            \end{enumerate}
            \item 农牧结合,良性循环,有利于发展“循环农业”实现农业生态系统的可持续发展
        \end{enumerate}
        \item 社会
        \begin{enumerate}
            \item 增产:维护“粮食安全”确保社会稳定
            \item 改善品质:改善农产品品质(如通过轮作可增加稻米蛋白质含量),尤其是有利于生产“健康食品”,有利于提高人民生活质量和营养保健水平
            \item 丰富种类:通过轮作增加农产品的“花色”和品种有利于改善膳食结构,提高健康水平
        \end{enumerate}
    \end{enumerate}
    \subsection{垄作栽培的优点}
    \begin{enumerate}
        \item 增大受光面积,光照条件好
        \item 起垄载培利于(旱季)灌溉和(雨季)排水
        \item 地温提高快,昼夜温差增大
        \item 起垄栽培后土壤通气性强(土质疏松)
        \item 加厚土壤层(增加表土厚度)
        \item 低温时期,垄沟注水可喊少垄上作物夜间受低温冻害影响
        \item 垄台能阻风和降低风速;利于集中施肥
    \end{enumerate}
    \subsection{病虫害少的原因}
    \begin{enumerate}
        \item 虫源
        \begin{enumerate}
            \item 环境封闭,不易受外来病虫侵袭
        \end{enumerate}
        \item 气候
        \begin{enumerate}
            \item 气温
            \begin{enumerate}
                \item 冬季低温(包括寒潮),春温低(积雪厚),夏季气温低或昼夜温差大,夜间气温低,不利于病虫繁殖
            \end{enumerate}
            \item 紫外线
            \begin{enumerate}
                \item 紫外线强或照射时间长,杀灭病虫害
            \end{enumerate}
            \item 干旱少雨的气候环境
        \end{enumerate}
        \item 水土
        \begin{enumerate}
            \item 水分,土壞条件好,作物生长发育好,抵抗病虫害力强
        \end{enumerate}
        \item 科技
        \begin{enumerate}
            \item 培育优良的抗病虫害品种
            \item 先进的病虫害防治技木
        \end{enumerate}
        \item 生态环境
        \begin{enumerate}
            \item 生态平衡,病虫天敌数量适宜
        \end{enumerate}
        \item 其他
        \begin{enumerate}
            \item 如森林火灾会烧死部分害虫和虫卵
        \end{enumerate}
    \end{enumerate}
    \subsection{农产品品质好}
    \begin{enumerate}
        \item 环境(气候,土壤,水源)质量好
        \item 昼夜温差
        \begin{enumerate}
            \item 有机质积累量(白天气温高,有利于有机质和糖分的合成,夜晚气温低可减少消耗)
        \end{enumerate}
        \item 光照:光合作用
        \item 气温高低
        \begin{enumerate}
            \item 生育期长短(农作物生长期越长,积累的有机质戏多,品质越好)
        \end{enumerate}
        \item 人类活动多少
        \begin{enumerate}
            \item (农业,工业,交通等)污染物排放量(开发历史短,人类活动少,在原生态环境中,农产品品质更优)
        \end{enumerate}
        \item 病虫害多少影响化肥和农药施用量(精准施肥提高品质)
        \item 农业技术(改良品种,改进耕作技术)
    \end{enumerate}
    \subsection{农作物单产高}
    \begin{enumerate}
        \item 自然条件
        \begin{enumerate}
            \item 生育期长,光照强,昼夜温差大,水分条件好,土壤肥沃
        \end{enumerate}
        \item 自然灾害少
        \begin{enumerate}
            \item 旱涝,风沙,盐碱,低温阴雨冻害,病虫害等较少
        \end{enumerate}
        \item 精细管理
        \begin{enumerate}
            \item 精细化管理,农业科技平高
        \end{enumerate}
        \item 农作物品种
    \end{enumerate}
    \subsection{农作物商品率高的原因}
    \begin{enumerate}
        \item 用于出售的衣产品产量占总产量的比重
        \item 产量高低
        \begin{enumerate}
            \item 自然条优劣,自然灾害多少,农业技术水平高低,人均耕地面积,机械化程度
        \end{enumerate}
        \item 用年多少
        \begin{enumerate}
            \item 人口数量多少
            \item 农产品加工工业发展
        \end{enumerate}
        \item 其他因素
        \begin{enumerate}
            \item 市场需求大小,农产品运输要求及运输条件(有些农产品易腐烂变质,仓储与运输过程中需要保险,进而影响运输成本,影响产品价格)
        \end{enumerate}
    \end{enumerate}
    \subsection{是否扩大种植规模}
    \begin{enumerate}
        \item 赞同
        \begin{enumerate}
            \item 有利条件
            \begin{enumerate}
                \item 自然条件优越,市场需求大
                \item 资源丰富
                \item 经验技术
            \end{enumerate}
            \item 经济意义
            \begin{enumerate}
                \item 推动当地经济发展
                \item 提高产量,获得规模效益
            \end{enumerate}
            \item 社会意义
            \begin{enumerate}
                \item 增加就业机会
                \item 促进基础设施建设
            \end{enumerate}
        \end{enumerate}
        \item 不赞同
        \begin{enumerate}
            \item 不利条件
            \begin{enumerate}
                \item 山地多,平原少,土地资源有限
                \item 气候干旱,会加剧水资源紧张状况
            \end{enumerate}
            \item 经济效益下降
            \begin{enumerate}
                \item 产量大增,可能会导致价格降低,产品积压
            \end{enumerate}
            \item 生态破坏
            \begin{enumerate}
                \item 荒漠化,水土流失,红漠化,石漠化,土壤盐碱化等
            \end{enumerate}
            \item 环境污染
            \begin{enumerate}
                \item 土壤,水污染等
            \end{enumerate}
        \end{enumerate}
    \end{enumerate}
    \subsection{农业区位条件的改造}
        \begin{tabular}{|c|c|}
            \hline
            限制条件  & 改造措施    \\
            \hline
            热量不足  &  \makecell[c]{用塑料大棚,玻璃温室\\地温低时可用盆栽并用支架支起}   \\
            \hline
            光照不足  & \makecell[c]{室内用日光灯,反光镜\\苹果树下用反光纸等}    \\
            \hline
            水源不足  & \makecell[c]{用日光温室改善热量条件,调节蒸发\\发展节水农业,耐旱农业\\采用滴灌,喷灌技术\\适当抽取地下水等}    \\
            \hline
            地形不利  &  \makecell[c]{改造地形,发展梯田\\选择河谷平原,河漫滩等地}   \\
            \hline
            红壤  &  \makecell[c]{掺沙\\补充熟石灰(草木灰),增施有机肥\\种茶树或松树}    \\
            \hline
            盐碱地  & \makecell[c]{调控水盐运动\\引淡淋盐,井排井灌,农田覆盖\\采用鱼塘台田模式}   \\
            \hline
            冻害  & \makecell[c]{秋冬季节,北方用人造烟雾\\利用秸秆,地膜覆盖等\\通过浇水或增施有机肥防霜冻}  \\
            \hline
            提高作物品质 & \makecell[c]{铺沙或鹅卵石,增加昼夜温差\\延长光照时间\\延长作物生长期\\采用良种\\提高科技含量}    \\
            \hline
        \end{tabular}
    \subsection{农业覆盖技术——覆膜的作用}
    \begin{enumerate}
        \item 光照(反光,透光率,时长)
        \begin{enumerate}
            \item 增加光效应,提高果品着色度
            \item 减少病虫害
            \item 抑制杂草生长
        \end{enumerate}
        \item 温度(热量,温差)
        \begin{enumerate}
            \item 调节地温:冬季保温,夏季降温
            \item 调整作物上市时间
        \end{enumerate}
        \item 土壤肥力(流水风力侵蚀)
        \begin{enumerate}
            \item 保土保肥,防止水土流失
        \end{enumerate}
        \item 水分(减少蒸发)
        \begin{enumerate}
            \item 增湿保墒
            \item 减轻次生盐渍化
        \end{enumerate}
        \item 压草的持性
        \begin{enumerate}
            \item 抑制杂草生长
        \end{enumerate}
    \end{enumerate}
    \subsection{农业覆盖技术——覆草类的作用}
    \begin{enumerate}
        \item 减少表层土壤水分蒸发
        \begin{enumerate}
            \item 保湿
            \item 减轻土壤次生盐碱化
        \end{enumerate}
        \item 表面粗糙,减少风力(流水)对土壤侵蚀
        \begin{enumerate}
            \item 保土保肥
        \end{enumerate}
        \item 滞留地表径流,增加下渗
        \begin{enumerate}
            \item 增湿
            \item 减轻土壤次生盐碱化
        \end{enumerate}
        \item 减少土壤与外界的热交换
        \begin{enumerate}
            \item 调节地温,减少土壤辐射
        \end{enumerate}
        \item 覆草减少地面光照
        \begin{enumerate}
            \item 抑制杂草丛生
            \item 避免幼苗受到烈日灼伤
        \end{enumerate}
        \item 腐烂后进入土壤
        \begin{enumerate}
            \item 提高土壤的肥力
            \item 引发病虫害
        \end{enumerate}
    \end{enumerate}
    \subsection{农业覆盖技术——覆砂砾的作用}
    \begin{enumerate}
        \item 光照(反光)
        \begin{enumerate}
            \item 有利于果实着色
        \end{enumerate}
        \item 温度(热量,温差(比热容))
        \begin{enumerate}
            \item 增大昼夜温差,利于有机质,糖分积累
            \item 保温(冬),降温(夏)
        \end{enumerate}
        \item 土壤肥力(表面粗糙减少侵蚀)
        \begin{enumerate}
            \item 保土保肥,保持水士
        \end{enumerate}
        \item 水分(下渗强蒸发弱)
        \begin{enumerate}
            \item 增湿保墒
            \item 减轻次生盐渍化
        \end{enumerate}
        \item 压草
        \begin{enumerate}
            \item 抑制杂草
        \end{enumerate}
    \end{enumerate}
    \subsection{农业可持续发展的措施}
    \begin{enumerate}
        \item 经济可持续
        \begin{enumerate}
            \item 因地制宜,合理布局,发展特色农业或建设商品稂生产基地
            \item 调整产业结构,发展多种经营
            \item 搞好农产品深加工,延长产业链,提高产品附加值
            \item 加大科技投入,培育新品种,提高产品质量
            \item 扩大销售渠道,积极开拓市场,树立品牌意识
            \item 推进农业的区域化,专业化,产业化,发农产品加工业,延长产业链
        \end{enumerate}
        \item 社会可持续
        \begin{enumerate}
            \item 加强农业基础设施建设,改善农业生产条件
            \item 国家政策,资金扶持,加水利等农业基础设施建设
        \end{enumerate}
        \item 生态可持续
        \begin{enumerate}
            \item 治理污染,注意生态环境保护,发展生态农业
            \item 保护性的发展农业,如发展观光农业,休闲农业
        \end{enumerate}
    \end{enumerate}
    \subsection{农业生产地区专门化的优点}
    \begin{enumerate}
        \item 充分发挥自然资源优势
        \item 更好地应用现代农业科学技术
        \item 提高农业劳动生产率
    \end{enumerate}
    \subsection{亚洲水稻种植业的特点,问题与措施}
    \begin{enumerate}
        \item 特点
        \begin{enumerate}
            \item 经营模式:小农经营
            \item 单位面积产量:高
            \item 机械化水平:低(日本除外,小型机械化)
            \item 水利工程量:大(季风区降水不稳定,水旱频繁)
            \item 科技水平:低
            \item 商品率:低
        \end{enumerate}
        \item 问题与措施\\
            \begin{tabular}{|l|l|}
                \hline
                问题  &  措施    \\
                \hline
                商品率低,农业生产结构单一   &   调整农业生产方式与结构,发展多种经营 \\ & 农林牧副渔业综合发展    \\
                \hline
                劳动生产率和商品率低,农业增收困难  & 发展农产品系列加工 \\ & 控制人口增长,适度扩大生产规模 \\ & 大力发展第二,第三产业和乡镇企业    \\
                \hline
                农业生产技术水平低   &  加大科技投入,发展优质,高产,高效农业    \\
                \hline
            \end{tabular}
    \end{enumerate}
    \subsection{大牧场放牧业的生产特点,区位条件,发展措施}
    \begin{enumerate}
        \item 特点
        \begin{enumerate}
            \item 商品率高
            \item 生产规模大
            \item 专业化程度高
        \end{enumerate}
        \item 区位条件
        \begin{enumerate}
            \item 气候温和,草类茂盛
            \item 地广人稀,土地租金低
            \item 距海港近,交通方便
            \item 历史悠久,市场广阔,商品率高
        \end{enumerate}
        \item 措施\\
            \begin{tabular}{|c|c|}
                \hline
                阿根廷的措施    &   对我国畜牧业发展的借鉴意义  \\
                \hline
                围栏放牧,划区轮牧   &   合理利用草场,使草场不退化   \\
                \hline
                \makecell[l]{开辟水源 \\ 种植牧草}  & 保护草场,开辟水源,建立人工草场,提高草场载畜量   \\
                \hline
                改善交通运输条件    &   发展交通和冷藏,保鲜技术    \\
                \hline
                培育良种牛,加强牛群病害研究 &   加大科技投入,培育良种   \\
                \hline
            \end{tabular}
    \end{enumerate}
    \subsection{乳畜业的生产特点,区位条件}
    \begin{enumerate}
        \item 特点
        \begin{enumerate}
            \item 面向城市市场,位于郊区靠近城市,商品率高
            \item 集约化程度高
            \item 机械化程度高
        \end{enumerate}
        \item 区位条件
        \begin{enumerate}
            \item 优质的多汁牧草,精饲料作物种植,广阔的市场需求
        \end{enumerate}
    \end{enumerate}
    \subsection{混合农业的区位条件,优点}
    \begin{enumerate}
        \item 区位条件
        \begin{enumerate}
            \item 气候温和适宜小麦和牧草生长
            \item 盆地地形,地势较平坦,土壤肥沃
            \item 地广人稀,可以发展大规模农场
            \item 临近海港,交通便利,市场广阔
            \item 经济发达,科技先进,机械化水平高
        \end{enumerate}
        \item 优点
        \begin{enumerate}
            \item 良性的农业生态系统(交替种植小麦,牧草或休耕有利于保持土壤肥力)
            \item 有效合理安排农事(小麦耕作活动与牧羊活动交替进行)
            \item 灵活性和市场适应性
        \end{enumerate}
    \end{enumerate}
    \subsection{工业区位}
    \begin{enumerate}
        \item 自然因素
        \begin{enumerate}
            \item 地理位置
            \item 土地(数量,质量,价格)
            \item 水源(数量,质量)
        \end{enumerate}
        \item 社会经济因素
        \begin{enumerate}
            \item 原料(数量,质量,种类,距离,运输成本)
            \item 动力(燃料)
            \item 市场(规模,潜力,距离,竞争力)
            \item 交通(便利,价格)
            \item 劳动力(数量,素质,价格,生活习惯)
            \item 政策(补贴,法规,税收,用地,服务水平)
            \item 科技(科研,人才)
            \item 集聚
            \item 工业基础,基础设施,产业协作
            \item 环境(环境问,污染影响,环境容量,处理技术)
        \end{enumerate}
    \end{enumerate}
    \subsection{大气污染工业布局}
    \begin{enumerate}
        \item 若该地有多方来风时,污染工业应布局在最小风频的上风
        \item 若本地有相对两侧来风时(如季风气候的杭州和孟买),污染工业布局在盛行风垂直的郊外
        \item 若该为单一风向时,污染工业布局在盛行风的下风向
    \end{enumerate}
    \subsection{工业集聚的优点与缺点}
    \begin{enumerate}
        \item 优点
        \begin{enumerate}
            \item 工业企业靠近其它经济活动,有利于共享基础设施和公共服务
            \item 生产上具有联系的企业集聚,一方面还能节约运输成本,降低能源消耗,集中处理废弃物,另一方面开展协作,促进技术创新,提高资源利用率
        \end{enumerate}
        \item 缺点
        \begin{enumerate}
            \item 用地紧张,地价上涨
            \item 水电供应不足
            \item 原料,燃料困难等问题
            \item 交通堵塞
            \item 污染加剧
            \item 劳动力短缺
        \end{enumerate}
    \end{enumerate}
    \subsection{工业转移的影响}
    \begin{enumerate}
        \item 迁岀地
        \begin{enumerate}
            \item 利
            \begin{enumerate}
                \item 改善环境
                \item 缓解能源,土地资源压力
                \item 利于产业结构优化升级
            \end{enumerate}
            \item 不利
            \begin{enumerate}
                \item 就业紧张
            \end{enumerate}
        \end{enumerate}
        \item 迁入地
        \begin{enumerate}
            \item 利
            \begin{enumerate}
                \item 促进当地经济发展,资源优势转变经济优势
                \item 增加就业机会,提髙居民收入
                \item 加快工业化,城市化进程
            \end{enumerate}
            \item 不利
            \begin{enumerate}
                \item 环境污染
            \end{enumerate}
        \end{enumerate}
    \end{enumerate}
    \subsection{鲁尔区的区位条件}
    \begin{enumerate}
        \item 丰富的煤炭资源
        \item 距离铁矿区较近(法国洛林,瑞典与俄罗斯经鹿特丹运入)
        \item 水源充沛
        \item 水陆交通便捷
        \item 市场广阔
    \end{enumerate}
    \subsection{鲁尔区衰落的原因}
    \begin{enumerate}
        \item 内因
        \begin{enumerate}
            \item 生产结构单一:煤炭和附铁工业是全区经济的基货
            \item 环境污染严重:新兴产业不愿来此安家落户
        \end{enumerate}
        \item 外因
        \begin{enumerate}
            \item 煤炭的能源地位下降
            \begin{enumerate}
                \item 石油,天然气的广泛使用
                \item 新技术炼钢的耗量逐渐降低
                \item 本区煤炭工业成本上升
            \end{enumerate}
            \item 世界性钢铁过剩
            \begin{enumerate}
                \item 产钢,出口钢的国家越来越多
                \item 经济危机,钢替代品的广泛使用
            \end{enumerate}
        \end{enumerate}
        \item 根本原因
        \begin{enumerate}
            \item 新技术革命的冲击
            \begin{enumerate}
                \item 产生了一大批新兴工业部门
                \item 对传统工业的生产和组织方式产生冲击
            \end{enumerate}
        \end{enumerate}
    \end{enumerate}
    \subsection{鲁尔区整治的措施}
    \begin{enumerate}
        \item 调整产业结构
        \begin{enumerate}
            \item 改造煤炭和钢铁工业,通过合并扩大企业规模
            \item 发展新兴工业和第三产业,促进经济结构多样化
        \end{enumerate}
        \item 调整工业布局
        \begin{enumerate}
            \item 采取平衡策略,新企业安排在核心地区的边缘地带
        \end{enumerate}
        \item 拓展交通,完善交通网
        \item 发展科技,繁荣经济
        \item 消除污染,美化环境
    \end{enumerate}
    \subsection{工业可持续发展的措施}
    \begin{enumerate}
        \item 调整产业结构,发展新兴工业(高新技术产业)和第三产业
        \item 调整工业布局,寻找最优区位
        \item 加强宣传,培育品牌,完善销售粱道,多元营销,扩大市场
        \item 延长产业链,加强技术创新,研发设计,提高自动化水平
        \item 培养,引进专业人才,提高劳动力素质
        \item 拓展交通,完善基础设施,提高区域开放程度
        \item 消除污染,美化环境,清洁生产,循环经济
    \end{enumerate}
    \subsection{五种交通运输方式的优缺点}
        \begin{tabular}{|c|l|l|}
            \hline
            方式   &   优点   &  缺点    \\
            \hline
            铁路   &   \makecell[l]{运量大速度快,运费较低 \\ 受自然因素影响小,连续性好}  &  \makecell[l]{造价高,消耗金属材料多 \\ 占地广,短途成本高} \\
            \hline
            公路   &    \makecell[l]{机动灵活,周转速度快,装卸方便 \\ 对各种自然条件适应性强} &  \makecell[l]{运量小,耗能多 \\ 成本高,运费较贵}   \\
            \hline
            水路   &   运量大,投资少,成本低    &    速度慢,灵活性和连续性差   \\
            \hline
            航空   &    速度快,运输效率高   &   运量小,能耗大,运费高 \\ & 设备投资大,技术要求高  \\
            \hline
            管道   &  运具与线路合二为一 \\ & 运量大,损耗小,平稳安全 \\ & 连续性强,管理方便   &    要铺设专门管道,设备投资大 \\ & 灵活性差   \\
            \hline
        \end{tabular}
    \subsection{集装箱的优点}
    \begin{enumerate}
        \item 节省包装和仓库费用
        \item 便于实现装卸作业的机械化
        \item 保证货物在运输过程中的安全
    \end{enumerate}
    \subsection{交通线的选线原则(低高大小)}
    \begin{enumerate}
        \item 成本要低,安全性要高,经济效益要大,环境破坏要小
    \end{enumerate}
    \subsection{以桥带路的原因及影响}
    \begin{enumerate}
        \item 山区
        \begin{enumerate}
            \item 地形起伏大,以桥代路可以减小坡度,降低输危险
            \item 缩短运距,节约运输时闽和成本
            \item 减少洪水,泥石流等自然灾害的破坏
        \end{enumerate}
        \item 跨河,湖,海
        \begin{enumerate}
            \item 缩短选距
        \end{enumerate}
        \item 冻土区(高海拔高纬度)
        \begin{enumerate}
            \item 防止地基塌陷;为生物迁徙留出通道
        \end{enumerate}
        \item 平原
        \begin{enumerate}
            \item 保护耕地
            \item 防止阻碍原有交通,保障交通通畅
        \end{enumerate}
        \item 不利影响
        \begin{enumerate}
            \item 技术要求高
            \item 成本增加
        \end{enumerate}
    \end{enumerate}
    \subsection{大桥,铁路建设难度大的自然因素}
    \begin{enumerate}
        \item 水文(洋流,潮汐,海水盐度,径流变化,水域宽度深度)
        \begin{enumerate}
            \item 洋流经过,潮汐运动方向和流速大
            \item 海水盐度高,易使桥墩被腐蚀
            \item 雨季河流径流量大,河水暴涨
            \item 水域宽度,深度等
        \end{enumerate}
        \item 生物(多样性,生态脆弱)
        \begin{enumerate}
            \item 生物多样性(保护/专门通道)
            \item 生态环境脆弱
            \item 生物对人类的危害
        \end{enumerate}
    \end{enumerate}
    \subsection{河运区位条件}
    \begin{enumerate}
        \item 自然条件
        \begin{enumerate}
            \item 水速:地形
            \item 水量,水位变化,结冰期:气候
            \item 含沙量:植被
            \item 河流长度,支流,流域面积:水系
        \end{enumerate}
        \item 社会经济条件
        \begin{enumerate}
            \item 流域内的经济,人口,交通网络联运,经济腹地
        \end{enumerate}
    \end{enumerate}
    \subsection{海运区位条件}
    \begin{enumerate}
        \item 水位深浅
        \item 航线长度
        \item 通航时间(结冰)
        \item 航道安全性(风浪洋流,潮汐,海雾,海冰,冰山,暗礁,海盗,政局)
        \item 沿岸港口(港口数量和吞吐能力(装卸,靠泊,仓储,集疏和物资供应能力))
        \item 技术
        \item 政策(沿途各国的经济政策,航运政策)
        \item 运输需求量,方向(沿岸城市多少,经济发展水平,货物的运输方向和运量(大宗物资))
    \end{enumerate}
    \subsection{港口的区位条件}
    \begin{enumerate}
        \item 自然条件(决定港口位置)
        \begin{enumerate}
            \item 水域条件,港阔水深港湾不淤不冻(航行条件和停泊条件)
            \begin{enumerate}
                \item 河港:沿河,水深(不淤积),流缓,河宽(提供淡水和空间)
                \item 海港:沿海,水深,易靠岸,有避风浪的海湾
            \end{enumerate}
            \item 筑港条件
            \item 陆地地质稳定,地形干坦,坡度适当(有利于安排建筑用地,港口设备)
        \end{enumerate}
        \item 社会经济条件(决定港口兴衰)
        \begin{enumerate}
            \item 经济腹地条件
            \begin{enumerate}
                \item 经济腹地是否广阔,客货流量大小,腹地经济性质
                \item 经济腹地约广阔与否影响着客货流量
                \item 客货流量影响着港口的兴衰
                \item 腹地经济性质决定港口性质(综合港,专业港等)
            \end{enumerate}
            \item 城市依托(交通便利,基础设施完善)
            \begin{enumerate}
                \item 为港口提供人力物力财力的支持
            \end{enumerate}
            \item 政策条件
            \begin{enumerate}
                \item 对外开放地区建成自由贸易港
            \end{enumerate}
        \end{enumerate}
    \end{enumerate}
    \subsection{航空港的区位条件}
    \begin{enumerate}
        \item 自然条件
        \begin{enumerate}
            \item 地形
            \begin{enumerate}
                \item 平坦开阔,利于跑道建设
                \item 坡度适当,保证排水
            \end{enumerate}
            \item 地质
            \begin{enumerate}
                \item 有良好的地质条件,地基要稳,坚实
            \end{enumerate}
            \item 气候
            \begin{enumerate}
                \item 少云雾,大风,暴雨天气日数
            \end{enumerate}
            \item 风向
            \begin{enumerate}
                \item 跑道沿盛行风的方向修建,利于飞机逆风起飞和降落
            \end{enumerate}
        \end{enumerate}
        \item 社会经济因素
        \begin{enumerate}
            \item 交通
            \begin{enumerate}
                \item 要与市内有便利的交通联系
            \end{enumerate}
            \item 经济
            \begin{enumerate}
                \item 要建在经济发达的地区
            \end{enumerate}
            \item 环境
            \begin{enumerate}
                \item 航空港噪音较大,与城市应有一定的距离
            \end{enumerate}
        \end{enumerate}
    \end{enumerate}
    \subsection{森林的环境效益}% do_on
        \begin{tabular}{|c|c|}
            \hline
            地区    &   森林的环境效益  \\
            \hline
            山地,丘陵  &   涵养水源,保持水土  \\
            \hline
            较干早地区  &   防风固沙,保护农田  \\
            \hline
            城市    &   美化环境,减弱噪音,调节气候    \\
            \hline
            交通线两侧  &   美化环境,减弱噪音,吸烟滞尘    \\
            \hline
            热带雨林分布区  &   \makecell[c]{维持全球碳氧平衡,保护生物多样性\\调节大气成分,促进水循环等} \\
            \hline
        \end{tabular}
    \subsection{三大区域发展阶段名称与空间结构特点}
        \begin{tabular}{|p{20mm}|l|}
            \hline
            区域发展阶段    &   空间结构特点    \\
            \hline
            以传统农业为主体的发展阶段  &   \makecell[l]{内部差异比较小,交通线数量少,分布散 \\ 区域对外开放程度最低,对外贸易规模小 \\ 突出表现为自给自足}  \\
            \hline
            工业化阶段  &  \makecell[l]{快速发展带动了区域的发展 \\ 出现一系列大规校的中心城市和工业基地 \\ 交通运输建设显著加快,区域对外开放程度逐步提高} \\
            \hline
            高效益的综合发展阶段    &   \makecell[l]{现代化的交通运输网络和信息商务网络逐步形成 \\ 区域内部的发展差异逐渐缩小,区域的开放程度和对外联系大幅度增强}    \\
            \hline
        \end{tabular}
    \subsection{三大自然区名称及界线}
        \begin{tabular}{|l|l|}
            \hline
            名称    &   界线    \\
            \hline
            东部季风区  &   大兴安岭以东,内蒙古高原以南,青藏高原东部边缘以东的广大地区    \\
            \hline
            西北内陆干旱半干早区    &   大兴安岭,贺兰山以西,昆仑山脉,祁连山脉以北的非季风区   \\
            \hline
            青藏高寒区  &   横断山脉以西,喜马拉雅山以北,昆仑山和阿尔金山以南地区    \\
            \hline
        \end{tabular}
    \subsection{四大地区的范围(中部)}
    \begin{enumerate}
        \item 晋,豫,鄂,皖,湘,赣
    \end{enumerate}
    \subsection{西部大开发的措施}
    \begin{enumerate}
        \item 抓好基础设施和生态环境建设
        \item 加快科教发展和人才开发推进科技创新
        \item 充分发挥地区的优势条件,积极发展特色产业
        \item 加大对外开放力度,积极发展对外贸易
        \item 巩因和发展农业基础
        \item 国家加大对中西部地区的政策支持
    \end{enumerate}
    \subsection{南水北调三条线路及优缺点}
    \begin{enumerate}
        \item 东线
        \begin{enumerate}
            \item 优点
            \begin{enumerate}
                \item 可调水量大,工程量小,投资少
            \end{enumerate}
            \item 缺点
            \begin{enumerate}
                \item 水质较差,逐级提水,运转费用高
            \end{enumerate}
            \item 生态环境问题
            \begin{enumerate}
                \item 长江口地区的影响,导致北方灌区土壤次生盐渍化等
            \end{enumerate}
        \end{enumerate}
        \item 中线
        \begin{enumerate}
            \item 优点
            \begin{enumerate}
                \item 水质较好,自流输水,运转费用低等
            \end{enumerate}
            \item 缺点
            \begin{enumerate}
                \item 工程量大,投资多,移民量大等
            \end{enumerate}
        \end{enumerate}
        \item 西线
        \begin{enumerate}
            \item 优点
            \begin{enumerate}
                \item 引水的水源点多,水质好
            \end{enumerate}
            \item 缺点
            \begin{enumerate}
                \item 地形,地质条件复杂,线路全为隧洞,工程技术难度大
            \end{enumerate}
            \item 生态环境问题
            \begin{enumerate}
                \item 生态环境脆弱,极易造成植被破坏,水土流失等问题
            \end{enumerate}
        \end{enumerate}
    \end{enumerate}
    \subsection{南水北调的影响}
    \begin{enumerate}
        \item 调出区
        \begin{enumerate}
            \item 利
            \begin{enumerate}
                \item 东线:流量减小水质下降
            \end{enumerate}
            \item 不利
            \begin{enumerate}
                \item 东线
                \begin{enumerate}
                    \item 泥沙淤积加重,航道淤塞
                    \item 河口海水上溯,盐度升高,水质下降
                    \item 生物多样性械少
                \end{enumerate}
                \item 中线:移民投资大
            \end{enumerate}
        \end{enumerate}
        \item 调入区
        \begin{enumerate}
            \item 利
            \begin{enumerate}
                \item 缓解水资源紧缺状况
                \item 保护湿地和生物多样性
                \item 促进经济发展
                \item 缓解地下水位下降和地面下沉状况
            \end{enumerate}
            \item 不利
            \begin{enumerate}
                \item 易造成土壤次生盐碱化
            \end{enumerate}
        \end{enumerate}
    \end{enumerate}
    \subsection{跨区域资源调配线路选择的影响因素}
    \begin{enumerate}
        \item 稳定的资源供应和市场消费能力
        \item 新建线路短(原有线路,建设距离),建设成本低。
        \item 施工难度小(地形地质,河流,冻土)
        \item 安全系数高(灾害,政局)
        \item 对沿线自然环境的影响小
    \end{enumerate}
    \subsection{西气东输的线路和意义}
    \begin{enumerate}
        \item 西部
        \begin{enumerate}
            \item 利
            \begin{enumerate}
                \item 资源优势转变成经济优势
                \item 推动基础设施建设,拉动相关产业发展增加就业
                \item 解决生活燃料,减少对植被破坏,改善生态环境
            \end{enumerate}
            \item 不利
            \begin{enumerate}
                \item 建设区植被破坏,荒漠化,生态环境退化
                \item 破坏文物古迹和雅丹地貌
            \end{enumerate}
        \end{enumerate}
        \item 东部
        \begin{enumerate}
            \item 利
            \begin{enumerate}
                \item 缓解能源紧张,改善能源结构,减轻能源运输压力
                \item 改善大气环境质量
                \item 带动城镇基础设施建设
            \end{enumerate}
        \end{enumerate}
    \end{enumerate}
    \subsection{西电东送的三条线路和影响}
    \begin{enumerate}
        \item 三条线路\\
            \begin{tabular}{|c|c|}
                \hline
                调出区  &   调入区  \\
                \hline
                \makecell[c]{黄河上游的水电\\晋,陕,内蒙的火电}  &   京津唐区    \\
                \hline
                长江上游干支流的水电    &   长江三角洲  \\
                \hline
                \makecell[c]{红水河流域的水电\\贵州的火电}  &   珠江三角洲  \\
                \hline
            \end{tabular}
        \item 影响
        \begin{enumerate}
            \item 输入地
            \begin{enumerate}
                \item 缓解能源紧张局面
                \item 促进经济发展
                \item 改变能原消费结构
                \item 减轻环境污染
                \item 减轻铁路运输压力
            \end{enumerate}
            \item 输出地
            \begin{enumerate}
                \item 把资源优势转化为经济优势
                \item 带动相关产业发展
                \item 增加就业机会
                \item 缩小东西部差距,对社会稳定起重要作用
            \end{enumerate}
        \end{enumerate}
    \end{enumerate}
    \subsection{影响产业转移的因素}
        \begin{tabular}{|p{20mm}|l|p{60mm}|}
            \hline
            转移方向 &   转移原因    &   影响    \\
            \hline
            沿海企业向内地的迁移  &  \makecell[l]{原材料价格 \\ 工资与地价水平 \\ 公用事业费用等方面的区域差异}  &  对生态环境造成不利的影响    \\
            \hline
            台湾企业向大陆转移  &  \makecell[l]{大陆经济的发展(投资环境的改善) \\ 廉价的劳动力 \\ 众多的发展机会 \\ 广阔的潜在消费市场}  &   \makecell[l]{加速了大陆劳动密集型产业发展\\加速大陆高技术产业的发展 \\ 创造了大量的就业机会} \\
            \hline
            广东边远地区的产业集群效应  &  \makecell[l]{人口稠密 \\ 交通拥堵 \\ 资本过剩 \\ 污染严重 \\ 白然资源不足}   &   缩小地区差别,实现区域内部的平衡 \\
            \hline
        \end{tabular}
    \subsection{荒漠化的成因及措施}
    \begin{enumerate}
        \item 自然原因
        \begin{enumerate}
            \item 气候干旱(基本条件)
            \item 地面疏松,多沙质沉积物(物质条件)
            \item 大风日数多且集中(动力条件)
            \item 气候异常(重要影响因素)
        \end{enumerate}
        \item 人为原因(主要原因)
        \begin{enumerate}
            \item 讨度樵采
            \item 过度放牧
            \item 过度开垦
            \item 水资源利用不当
            \begin{enumerate}
                \item 不合理灌溉:土壤盐碱化
                \item 草原地区打井数量过多:斑状荒漠化
                \item 上游过度引水灌溉:下游因缺水而使绿洲退化
            \end{enumerate}
            \item 工程建设
        \end{enumerate}
    \end{enumerate}
    \subsection{东北黑土流失的原因及措施}
    \begin{enumerate}
        \item 原因
        \begin{enumerate}
            \item 降水强度大,流水的侵蚀强
            \item 冬,春季风大,风蚀作用强
            \item 不合理的耕作方式
            \item 麻林砍伐,地表植被破坏,覆盖率低
            \item 过度开垦
            \item 不合理的开矿等
        \end{enumerate}
        \item 保护措施
        \begin{enumerate}
            \item 保护性耕作
            \begin{enumerate}
                \item 休耕,限耕,免耕,轮作等
            \end{enumerate}
            \item 水士保持
            \begin{enumerate}
                \item 坡面治理,沟壑整治,植树种草
            \end{enumerate}
            \item 土壤培肥
            \begin{enumerate}
                \item 秸秆还田,增施有机肥
            \end{enumerate}
        \end{enumerate}
    \end{enumerate}
    \subsection{东北森林覆盖率下降的成因及措施}
    \begin{enumerate}
        \item 问题
        \begin{enumerate}
            \item 超采严重
            \item 采育失调
            \item 植被覆盖率下降
            \item 生态失衡
        \end{enumerate}
        \item 措施
        \begin{enumerate}
            \item 采育结合,营造人工林
            \item 促进珍贵树种的更新
            \item 提高木材综合利用率
            \item 加强自然保护区建设
            \item 调整产业结构,多种经营
        \end{enumerate}
    \end{enumerate}
    \subsection{华北春旱的成因及措施}
    \begin{enumerate}
        \item 成因
        \begin{enumerate}
            \item 自然
            \begin{enumerate}
                \item 春季气温回升快,蒸发旺盛
                \item 多大风,加速水分蒸发
                \item 仍受冬季风影响,东南风带来得水汽沙,降水少
                \item 河流刚结束枯水期,径流量小
            \end{enumerate}
            \item 人为
            \begin{enumerate}
                \item 冬小麦返青,生长旺盛需水量大
                \item 人口稠密,工业生产和生活用水量增加
            \end{enumerate}
        \end{enumerate}
        \item 措施
        \begin{enumerate}
            \item 推广喷,滴技术,发展节水农业
            \item 建设水利工程,修建水库,跨流域调水
            \item 植树造林,涵养水源
            \item 节约用水,防治水污染
        \end{enumerate}
    \end{enumerate}
    \subsection{华北地面沉降的成因,危害及措施}
    \begin{enumerate}
        \item 成因
        \begin{enumerate}
            \item 自然原因
            \begin{enumerate}
                \item 构造升降运动,地震,火山活动,松软地基,地面加载(如冰川)
            \end{enumerate}
            \item 人为原因
            \begin{enumerate}
                \item 开釆地下水,油气资源,地面加载(如减市建设,水库)
            \end{enumerate}
        \end{enumerate}
        \item 危害
        \begin{enumerate}
            \item 洪涝灾害加剧
            \item 建筑物基础下沉,工程设施毁坏
            \item 导致沿海地区风暴及海浪侵蚀加剧
            \item 造成沿海地区海水倒灌,引起盐碱化
            \item 地下管道破环污水外溢
        \end{enumerate}
        \item 措施
        \begin{enumerate}
            \item 严格控制地下水开釆规模
            \item 雨季(汛期)回补地下水
            \item 节约用水
            \item 兴修水利,跨区域调水,以地表水代替地下水资源
            \item 油气田通过人工回灌,对抽汲的液体进行等体积替换等
        \end{enumerate}
    \end{enumerate}
    \subsection{华北沙尘暴的成因及措施}
    \begin{enumerate}
        \item 成因
        \begin{enumerate}
            \item 自然原因
            \begin{enumerate}
                \item 接近沙尘源地,沙源丰富
                \item 接近冬季风源地,春季多大风
                \item 春季降水少,升温快,蒸发旺盛,地表干燥裸露,植被覆盖率低
            \end{enumerate}
            \item 人为原因
            \begin{enumerate}
                \item 过度开垦,过度樵采
                \item 过度放牧,工矿交通建设
            \end{enumerate}
        \end{enumerate}
        \item 措施
        \begin{enumerate}
            \item 退耕还草,林
            \item 控制载畜量,人工草场,轮牧
            \item 文明施工
            \item 制定法律法规,加强管理
        \end{enumerate}
    \end{enumerate}
    \subsection{土壤盐碱化的成因及措施}
    \begin{enumerate}
        \item 成因
        \begin{enumerate}
            \item 自然原因
            \begin{enumerate}
                \item 气候
                    \item (气温与降水)干旱,降水少
                    \item 气温高,多大风天气,蒸发旺盛
                \item 地形
                    \item 地势低洼,排水不畅,地表水下渗,地下水位升高
                \item 地下水
                    \item 地下水位埋藏浅
                \item 土壤
                    \item 碱性土壤(地下水盐度高或海水倒灌)
            \end{enumerate}
            \item 人为原因
            \begin{enumerate}
                \item 不合理的灌溉
                    \item 大水浸灌,只灌不排,导致水在地表聚集,大量下渗,地下水位升高,盐分被带到地表
                \item 沿海地区过度抽取地下水
                    \item 引起海水入侵地下水,进而随着地下水上升,增加了土地盐分
                \item 兴修水利工程
                    \item 补给地下水,水位升高
            \end{enumerate}
        \end{enumerate}
        \item 措施
        \begin{enumerate}
            \item 引淡淋盐
            \item 井排井灌,建立现代化排水系统
            \item 节水灌溉技术:滴灌喷灌
            \item 地膜覆盖(抑制水分蒸发)
            \item 生物措施:种植耐盐碱作物
        \end{enumerate}
    \end{enumerate}
    \subsection{黄土高原水土流失的成因,措施}
    \begin{enumerate}
        \item 成因
        \begin{enumerate}
            \item 自然原因
            \begin{enumerate}
                \item 降水
                    \item 变率大,集中在夏季,多暴雨,冲刷作用强
                \item 地形
                    \item 从平原向地过渡,地形坡度大,坡面物质不稳定
                \item 植被
                    \item 从森林向草原过渡,植被覆盖率低,涵养水源能力差
                \item 土壤
                    \item 由粉沙颗粒组成,土质疏松,垂直节理发育,抗蚀能力低
            \end{enumerate}
            \item 人为原因
            \begin{enumerate}
                \item 毁林开荒
                \item 过度憔采
                \item 不合理的耕作制度(轮荒)
                \item 开矿
            \end{enumerate}
        \end{enumerate}
        \item 措施
        \begin{enumerate}
            \item 退耕还林还草
            \begin{enumerate}
                \item 压缩农业用地,建成旱涝保收高产稳产农田
                \item 扩大林草种植面积,因地制宜营造防护林,经济林,薪炭林用林
                \item 大开展土地复垦工作
            \end{enumerate}
            \item 调整土地利用结构
            \item 加强小流域的综合治理
            \begin{enumerate}
                \item 工程措施
                    \item 打坝淤地平整土地,修筑梯田
                \item 生物措施
                    \item 植树种草
                \item 农业技术措施
                    \item 科学拖肥,选育良种,地膜覆盖
            \end{enumerate}
            \item 开矿时要有计划的存放表土,大力开发复垦工作
        \end{enumerate}
    \end{enumerate}
    \subsection{南方低山丘陵红漠化的成因及措施}
    \begin{enumerate}
        \item 成因
        \begin{enumerate}
            \item 自然原因
            \begin{enumerate}
                \item 土壤
                    \item 红壤“酸,瘦,黏”,较贫瘠
                \item 气候
                    \item 降水丰沛,集中且多暴雨
                \item 地形
                    \item 低山丘陵为主,地表起伏大
            \end{enumerate}
            \item 人为原因
            \begin{enumerate}
                \item 人多地少耕地不足,毁林开荒
                \item 农村生活能源短缺,伐木取薪
            \end{enumerate}
        \end{enumerate}
        \item 措施
        \begin{enumerate}
            \item 农业资源综合开发
            \begin{enumerate}
                \item 发展立体农业,多种经营
                \item 调整农业结构,开发优势资源
            \end{enumerate}
            \item 加强生态建设
            \begin{enumerate}
                \item 封山育林,保持水土
                \item 解决农民的生活能源问题
                    \item 推广生活用煤
                    \item 推广节能灶
                    \item 沼气池
                    \item 营造速生薪炭林
                    \item 发展小水电
            \end{enumerate}
        \end{enumerate}
    \end{enumerate}
    \subsection{云贵高原石漠化的成因及措施}
    \begin{enumerate}
        \item 成因
        \begin{enumerate}
            \item 自然原因
            \begin{enumerate}
                \item 土壤
                    \item 石灰岩广布,成土慢,土层薄,土壤贫瘠
                \item 气候
                    \item 降水丰沛,集中且多暴雨
                \item 地形
                    \item 喀斯特地貌,地表起伏大
                \item 水文
                    \item 河流众多,水量丰富,流水侵蚀作用强
            \end{enumerate}
            \item 人为原因
            \begin{enumerate}
                \item 能派不足,过度樵采,毁林开荒,过度放牧,植被破坏严重
            \end{enumerate}
        \end{enumerate}
        \item 措施
        \begin{enumerate}
            \item 加强管理,封山育林退耕还林
            \item 砌墙卤坡,整修梯田
            \item 打坝淤地保持水士
            \item 生态移民
            \item 改良土壤
            \item 推广省柴灶,营造薪炭林,小水电
        \end{enumerate}
    \end{enumerate}
    \subsection{湿地的成因}
    \begin{enumerate}
        \item 来水多
        \begin{enumerate}
            \item 大气降水:降水丰富区
            \item 冰雪融水
            \begin{enumerate}
                \item 冰川融水:高纬,高山地区
                \item 积雪融水
            \end{enumerate}
            \item 河流水
            \begin{enumerate}
                \item 河水泛滥
                    \item 河网密布
                    \item 水位变化大
                    \item 凌讯
                    \item 河流交汇处,河流流速慢
                    \item 汇水面积大,水量大
                    \item 海水顶托,下泄不畅
            \end{enumerate}
            \item 地下水:地下水位较高
            \item 海水:沿海地区,海水顶托
        \end{enumerate}
        \item 去水少
        \begin{enumerate}
            \item 排水不畅:地势低洼(和周围其他地区相比)
            \item 蒸发弱
            \begin{enumerate}
                \item 气温低
                    \item 纬度高
                    \item 海拔高
            \end{enumerate}
            \item 下渗少
            \begin{enumerate}
                \item 有冻土
                \item 土质黏重
                \item 基岩分布
            \end{enumerate}
            \item 植被蓄水
            \begin{enumerate}
                \item 植被覆盖率高,涵养水源能力强
            \end{enumerate}
        \end{enumerate}
    \end{enumerate}
    \subsection{湿地的价值}
    \begin{enumerate}
        \item 生态效益
        \begin{enumerate}
            \item 涵养水源,调蓄洪水
            \item 净化水中污染物质
            \item 调节气候,美化环境
            \item 维持生物多样性
        \end{enumerate}
        \item 经济效益
        \begin{enumerate}
            \item 提供水资源
            \item 提供农副产品,工业原料
            \item 航运
        \end{enumerate}
        \item 社会效益
        \begin{enumerate}
            \item 旅游观光
            \item 教肓科研
        \end{enumerate}
    \end{enumerate}
    \subsection{湿地萎缩的原因及措施}
    \begin{enumerate}
        \item 原因
        \begin{enumerate}
            \item 自然原因
            \begin{enumerate}
                \item 沉积物自然充满湖沼
                \item 时间极其漫长
            \end{enumerate}
            \item 人为原因
            \begin{enumerate}
                \item 土壤侵蚀
                    \item 入流泥沙量大增
                \item 环境问题
                    \item 入流营养物增加,使湖沼内藻类与水草丛生
                \item 围湖(海)造田
                    \item 湖沼和海滨滩涂面积剧减,乃至消失
                \item 大量引水灌溉,河流截流改向
                    \item 水量减少,有些湖沼在几十年内就明显缩小,变浅直至完全干涸
            \end{enumerate}
        \end{enumerate}
        \item 措施
        \begin{enumerate}
            \item 恢复
            \begin{enumerate}
                \item 退耕还湿,植树造林,清淤疏泼
                \item 停止无序开发,合理规划,杜绝不合理占用
                \item 内陆湖泊:加强管理,统筹分配,提高水资源利用率
            \end{enumerate}
            \item 治理
            \begin{enumerate}
                \item 减少化肥农药使用(生态农业),控制养殖规模
                \item 关停重污染企业,污水处理排放,生态移民
                \item 修建引水工程,加快自净速度
            \end{enumerate}
            \item 开发
            \begin{enumerate}
                \item 适度发展养殖业
                \item 水产品加工业
                \item 旅游业
            \end{enumerate}
            \item 管理(政府)
            \begin{enumerate}
                \item 加强宣传,增强保护意识
                \item 设立自然保护区(湿地公园)
                \item 制定相关法規,加强管理
            \end{enumerate}
        \end{enumerate}
    \end{enumerate}
    \subsection{评价水利工程的影响}
    \begin{enumerate}
        \item 有利影响
        \begin{enumerate}
            \item 经济效益
            \begin{enumerate}
                \item 产生防洪,发电,航运,灌溉,养殖和旅游等综合经济效益
            \end{enumerate}
            \item 生态效益
            \begin{enumerate}
                \item 调节库区气候,缓解生态环城压力
                \item 拦截泥沙,降低河流含沙量,提高水质
            \end{enumerate}
            \item 社会效益
            \begin{enumerate}
                \item 降低洪涝威胁,保障人们生命财产安全
                \item 促进产业调整,推动库区经济发展
            \end{enumerate}
        \end{enumerate}
        \item 不利影响
        \begin{enumerate}
            \item 库区
            \begin{enumerate}
                \item 库区派积泥沙,库容减小
                \item 易诱发地质灾害,如地震
                \item 影响生物洄游,主物多样性减少
                \item 淹没文物古迹
                \item 水质变差
                \item 店区蓄水后地下水位上升,易导致土地盐碱化
            \end{enumerate}
            \item 下游
            \begin{enumerate}
                \item 地貌
                    \item 来水来沙臧少,三角洲萎缩,海水入侵,海岸线后退
                \item 气候
                    \item 降水减少
                \item 河流
                    \item 径流量减少,海水倒灌,水质变差
                \item 土壤
                    \item 海水倒灌土壤盐碱化加剧,泥沙沉积减少,土地肥力下降,农业减产
                \item 生物
                    \item 河口渔业资源减少
                    \item 湿地减少,生物多样性减少
            \end{enumerate}
        \end{enumerate}
    \end{enumerate}
    \subsection{拦河坝——高坝低坝的优缺点(水能开发条件的比较)}
    \begin{enumerate}
        \item 高坝
        \begin{enumerate}
            \item 优点
            \begin{enumerate}
                \item 库容大,调节洪水的能力强,发电水头高,有效发电时间保证率大
            \end{enumerate}
            \item 缺点
            \begin{enumerate}
                \item 工程量大,淹没面积大
                \item 建设周期长,设计难度大,对坝址地质条件要求高
            \end{enumerate}
        \end{enumerate}
        \item 低坝
        \begin{enumerate}
            \item 优点
            \begin{enumerate}
                \item 工程量小,淹没土地面积小,移民数量少
                \item 建设周期短,耗资少,蓄水量小,溃坝的威胁小
                \item 对流域生态环境影响小
            \end{enumerate}
            \item 缺点
            \begin{enumerate}
                \item 库容小,调节洪水的能力小
                \item 发电水头低,有效发电时间保证率小
            \end{enumerate}
        \end{enumerate}
    \end{enumerate}
    \subsection{抽水蓄能电站的优点}
    \begin{enumerate}
        \item 解决电力系统日益突出的调峰问题
        \item 它可将电网负荷低时的多余电能,转变为电网高峰时期的高价值电能
        \item 明显改善了局部电网电压偏高的状况,保证电网电压稳定
        \item 发挥事故备用作用,保障电力系统安全稳定运行
        \item 提高系统中火电站和核电站的效率,提高电力利用率
    \end{enumerate}
    \subsection{游荡型河道的形成条件及治理措施}
    \begin{enumerate}
        \item 条件
        \begin{enumerate}
            \item 来沙量多,强烈的泥沙堆积使河床不断抬高
            \item 比降较大
            \item 流量变幅大,洪峰暴涨暴落
            \item 河床边界物质抗冲蚀性弱,河床对水流的约束性差
        \end{enumerate}
        \item 治理措施
        \begin{enumerate}
            \item 采取综合治理措施,包括水土保持,修建水库,发展淤灌和河床整治
        \end{enumerate}
    \end{enumerate}
    \subsection{辫状水系的成因}
    \begin{enumerate}
        \item 流量很不稳定,河水暴涨暴落,且含沙量大
        \item 当洪峰进入山前倾斜平原时,流速减慢,泥沙大量堆积形成浅滩
        \item 洪峰退后,浅滩出露形成若干沙岛及多股的忽分忽合河道
        \item 洪水再次来时,在原有的河道中又会形成新的沙岛以及新的河道
    \end{enumerate}
    \subsection{水体结冰的影响因素}
    \begin{enumerate}
        \item 气温:低易结冰
        \item 水的流动性:弱,易结冰
        \item 水的盐度:低易结冰
        \item 水深及水域面积:浅,小,易结冰
        \item 风力
        \begin{enumerate}
            \item 薄冰时吹大风,不易结冰
            \item 厚冰时吹大风,加剧结冰
        \end{enumerate}
    \end{enumerate}
    \subsection{湖泊水量大小的影响因素}
    \begin{enumerate}
        \item 气候
        \item 流域面积
        \item 湖泊面积
        \item 植被,河流,湿地等调节
        \item 用水
    \end{enumerate}
    \subsection{湖泊风浪大小的影响因素}
    \begin{enumerate}
        \item 气压带风带,篮行凤
        \item 温差大小
        \item 是否有狭管效应(山谷地区湖泊)
        \item 湖陆风
        \item 注意:结冰不考虑风浪
    \end{enumerate}
    \subsection{外流河湖变成内流河湖的成因}
    \begin{enumerate}
        \item 水平衡因素
        \begin{enumerate}
            \item 气候干旱,降水量小于蒸发量\\(入湖径流减少,蒸发量逐渐增大,湖泊水位下降,无径流排出,形成内流湖)
        \end{enumerate}
        \item 地壳运动
        \begin{enumerate}
            \item 青海湖原来是外流湖,后因地壳运动形成内流湖
        \end{enumerate}
        \item 河道变迁
        \begin{enumerate}
            \item 乌裕尔河下游河道泥沙淤塞,形成内流湖
        \end{enumerate}
        \item 山区:冰川运动
        \begin{enumerate}
            \item 如冰蚀或冰碛物的堵塞形成水川湖:纳木镨湖
            \item 滑坡,泥石流,山体崩塌致使河道堵塞形成堰塞湖:羊卓雍措
        \end{enumerate}
    \end{enumerate}
    \subsection{盐湖的形成过程及特点}
    \begin{enumerate}
        \item 形成过程
        \begin{enumerate}
            \item 地壳运动导致岩层断裂下陷,地势下降,形成地势低注的盆地
            \item 然后盆地积水成湖
            \item 河流径流携带盐分不断汇入积累,且湖泊无径流流出排泄盐分
            \item 随着全球变暖气候干旱,蒸发量增大
            \item 盐度进一步增大,形成咸水湖,最终形成盐湖
        \end{enumerate}
        \item 特点
        \begin{enumerate}
            \item 封闭,蒸发量大,盐度大,面积小,降水少,水较浅,地势较平坦等
        \end{enumerate}
    \end{enumerate}
    \subsection{环境问题的分类及主要表现}
        \begin{tabular}{|c|c|}
            \hline
            分类    &   主要表现    \\
            \hline
            资源短缺    &   \makecell[c]{水资源短缺\\土地资源短缺\\能源短缺}  \\
            \hline
            生态破坏    &   \makecell[c]{森林的环境调节功能下降\\水土流失,土地荒漠化\\土地盐碱化\\生物多样性减少}   \\
            \hline
            环境污染    &   \makecell[c]{大气污染,水体污染,土壤污染\\固体废弃物污染\\噪声污染\\放射性污染\\海洋污染}  \\
            \hline
        \end{tabular}
    \subsection{土地资源的保护措施}
    \begin{enumerate}
        \item 制定相关法律法规(土地管理法),严禁乱占耕地
        \item 合理规划,提高土地资源利用率
        \begin{enumerate}
            \item 非农建设用地,节约
            \item 土地整理
            \begin{enumerate}
                \item 村落归并,退宅还耕
                \item 矿区复垦
            \end{enumerate}
        \end{enumerate}
        \item 提高耕地质量
        \begin{enumerate}
            \item 采取工程,生物措施减轻生态破坏
            \begin{enumerate}
                \item 植树造林
                \item 轮作,轮牧,休牧,休耕
                \item 梯田
                \item 保护性犁地(秸秆留茬,还田)
            \end{enumerate}
            \item 改良土壤,提高肥力,防治土壤污染,防治土地退化
        \end{enumerate}
    \end{enumerate}
    \subsection{土壤板结的原因及措施}
    \begin{enumerate}
        \item 原因
        \begin{enumerate}
            \item 大水漫灌
            \item 农田土壤质地粘重,耕作层浅
            \item 有机肥使用不足,秸秆还田量少
            \item 长期单一地偏施化肥
            \item 农耕措施不当导致土塅结构破坏
            \item 风沙,暴雨水土流失
            \item 塑料制品过多的投入
        \end{enumerate}
        \item 措施
        \begin{enumerate}
            \item 摻沙,增施有机肥,改变土壤的物理性状
            \item 秸秆还田
            \item 农家肥与无机肥结合,增施有机肥
            \item 注意适度深耕
            \item 减少或杜绝塑料制品的使用
        \end{enumerate}
    \end{enumerate}
    \subsection{草场退化的原因及措施}
    \begin{enumerate}
        \item 原因
        \begin{enumerate}
            \item 自然原因
            \begin{enumerate}
                \item 气候异常,降水较少
                \item 鼠害虫害
            \end{enumerate}
            \item 人为原因
            \begin{enumerate}
                \item 粗放经营
                \item 过度放牧
                \item 过度农垦
                \item 滥采滥挖
                \item 工业污染
            \end{enumerate}
        \end{enumerate}
        \item 措施
        \begin{enumerate}
            \item 退耕还草,封育草场,建立饲草料基地,舍饲养畜
            \item 实行禁牧,轮牧制度,控制载畜量
        \end{enumerate}
    \end{enumerate}
    \subsection{生物多样性减少的成因及措施}
    \begin{enumerate}
        \item 原因
        \begin{enumerate}
            \item 人为原因
            \begin{enumerate}
                \item 破坏生物栖息地,过度采集和捕猎
                \item 外来物种入侵,环境污染,生态破坏
            \end{enumerate}
        \end{enumerate}
        \item 措施
        \begin{enumerate}
            \item 就地保护
            \begin{enumerate}
                \item 建立自然保护区
            \end{enumerate}
            \item 迁地保护
            \begin{enumerate}
                \item 濒危物种移入适宜环境进行特殊保护和管理
            \end{enumerate}
            \item 建立濒危物种种子库,基因库,以保护遗传资源
            \item 采育结合,合理放牧,实行禁渔期制度
            \item 控制人口增长,防治环境污染
            \item 颁布相关的法律,法规,加大宣传教育
        \end{enumerate}
    \end{enumerate}
    \subsection{水体富营养化(水华,赤潮)的成因及措施}
    \begin{enumerate}
        \item 水华
        \begin{enumerate}
            \item 原因
            \begin{enumerate}
                \item 大量N,P等营养元素排放(工业废水,生活污水农业废水)
                \item 水体较封闭,水流缓慢,净化速度慢
                \item 适宜的温度:藻类繁殖(耗氧,毒,遮阳)
            \end{enumerate}
            \item 措施
            \begin{enumerate}
                \item 立法限制工业污染源的排放,清洁生产,实现工业污水“达标排放"
                \item 建设城市污水处理厂
                \item 农业
                \item 合理施肥
                \item 控制湖区畜禽,水产养殖规模
                \item 湖底清淤,调水加快净化
            \end{enumerate}
        \end{enumerate}
        \item 赤潮
        \begin{enumerate}
            \item 原因
            \begin{enumerate}
                \item 自然原因
                    \item 气温高
                    \item 静水
                    \item 静风
                    \item 海域相对封闭
                \item 人为原因
                    \item 沿岸地区人口稠密,经济发达,工业和生活污水多
                    \item 农业生产过程中大量使用化肥,农药
                    \item 水产养殖
            \end{enumerate}
            \item 措施
            \begin{enumerate}
                \item 严格执法,控制污水入海量
                \item 人工打捞
                \item 提高污水处理技术,治理污染
                \item 加强对赤潮的监测和预报,制定应急预案
                \item 沿岸地区加强合作,共同治理污染
            \end{enumerate}
        \end{enumerate}
    \end{enumerate}
    \subsection{水体重金属污染的原因及措施}
    \begin{enumerate}
        \item 原因
        \begin{enumerate}
            \item 含有汞,铅,镉,铬,砷等重金属的工业废水的排放
            \item 农业生产中,污水灌溉,含重金属的农药,劣质化肥等的不合理使用
            \item 垃圾渗滤液的泄漏,含铅汽油的使用
            \item 采矿,废石和尾矿随意堆放
        \end{enumerate}
        \item 措施
        \begin{enumerate}
            \item 工业污水净化处理,达标后排放
            \item 合理使用农药化肥
            \item 不乱扔垃圾,实现垃圾分类回收
            \item 对已造成的污染可采取物理吸附法,化学沉淀法,电解法,生物净化法等
        \end{enumerate}
    \end{enumerate}
    \subsection{海洋石油污染的成因,危害及措施}
    \begin{enumerate}
        \item 原因
        \begin{enumerate}
            \item 近海石油的开采,加工和运输,海上油轮泄漏。
            \item 轮船废油废水排放
            \item 陆地污水排放,工业垃圾倾倒
        \end{enumerate}
        \item 措施
        \begin{enumerate}
            \item 制定有关法规,严禁非法排放含油污水
            \item 监测监视海区石油污染状况,制定应急预案
            \item 采用围栏阻止石油扩散,然后再回收
            \item 喷洒低毒性的化学消油剂,如粉状石灰
            \item 投放微生物加快降解
        \end{enumerate}
    \end{enumerate}
    \subsection{大气污染——灰霾的成因及措施}
    \begin{enumerate}
        \item 原因
        \begin{enumerate}
            \item 是水平方向静风现象的增多
            \begin{enumerate}
                \item 城市建设的迅速发展,大楼越建越髙
                \item 增大了地面摩擦系数,使风流经城区时明显减弱
            \end{enumerate}
            \item 是垂直方向的逆温现象
            \item 悬浮颗粒物的增加
            \begin{enumerate}
                \item 汽车尾气
                \item 燃煤废气
                \item 工业废气
                \item 建筑工地和道路交通产生的扬尘
                \item 北方降水少,植被覆盖率低,沙尘多
            \end{enumerate}
        \end{enumerate}
        \item 措施
        \begin{enumerate}
            \item 严格立法与执法,提高大气质量标准
            \item 提高汽车尾气排放标准
            \item 积极发展煤炭的气化,液化技术
            \item 推广利用新能源,优化能源结构
            \item 提高植被覆盖率
        \end{enumerate}
    \end{enumerate}
    \subsection{光化学烟雾的成因及措施}
    \begin{enumerate}
        \item 原因
        \begin{enumerate}
            \item 汽车尾气排放量大
            \item 强烈的紫外线照射(氮氧化物和碳氢化物在紫外线作用下产生的浅蓝色烟雾)
            \item 气象
            \begin{enumerate}
                \item 逆温和不利于扩散的气象条件
            \end{enumerate}
            \item 地形
            \begin{enumerate}
                \item 如河谷,盆地地形
            \end{enumerate}
        \end{enumerate}
        \item 措施
        \begin{enumerate}
            \item 预防
            \begin{enumerate}
                \item 推广清洁汽车燃料,安装尾气净化装置
            \end{enumerate}
            \item 治理
            \begin{enumerate}
                \item 改善交通结构,发展公共交通,减少排放
            \end{enumerate}
            \item 管理
            \begin{enumerate}
                \item 制订法律法规
            \end{enumerate}
            \item 意识
            \begin{enumerate}
                \item 宣传教育,公众加入环保
            \end{enumerate}
        \end{enumerate}
    \end{enumerate}
    \subsection{臭氧层空洞的成因,危害,措施}
    \begin{enumerate}
        \item 原因
        \begin{enumerate}
            \item 氟氯烃化合物,通过光化学反应大量消耗臭氧
        \end{enumerate}
        \item 危害
        \begin{enumerate}
            \item 危害人体健康
            \begin{enumerate}
                \item 增加皮肤癌,主要是黑色素癌
                \item 损害眼睛,增加白内障患者
                \item 削弱免疫力,增加传染病患者
            \end{enumerate}
            \item 破坏生态环境和工农业生产
            \begin{enumerate}
                \item 农产品减少及其品质下降
                \item 减少渔业产量
                \item 使塑料等高分子聚合物加速老化
            \end{enumerate}
        \end{enumerate}
        \item 措施
        \begin{enumerate}
            \item 国际合作
            \item 积极研制新型的制冷系统,减少氟氯烃气体排放
        \end{enumerate}
    \end{enumerate}
    \subsection{酸雨的成因,危害,措施}
    \begin{enumerate}
        \item 原因
        \begin{enumerate}
            \item 燃烧煤,石油,天然气等,不断向大气中排放二氧化硫和氧化氮等酸性气体
        \end{enumerate}
        \item 危害
        \begin{enumerate}
            \item 使河湖水酸化,影响鱼类生长繁殖,乃至死亡
            \item 使土壤酸化,危害森林和农作物生长
            \item 腐蚀建筑物和文物古迹
            \item 危及人体健康
        \end{enumerate}
        \item 措施
        \begin{enumerate}
            \item 调整能源结构,开发利用新能源,清洁能源
            \item 调整工业布局(影响小,扩散快),产业结构(重化比重降低)
            \item 提高能源利用率(节约资源,减排)
            \item 发展洁净煤燃烧技术和煤炭脱硫技术,加强废气中SO2的回收与利用
            \item 循环经济,清洁生产
        \end{enumerate}
    \end{enumerate}
    \subsection{土壤污染的成因及措施}
    \begin{enumerate}
        \item 原因
        \begin{enumerate}
            \item 大量使用农药,化肥
            \item 长期使用污水灌溉
            \item 工业及生活污水任意排放以及生产生活中的固体废弃物在土壤上任意堆放,使其中的有毒有害物质进入土壤,使土壤性状发生改变
        \end{enumerate}
        \item 措施
        \begin{enumerate}
            \item 控制污染源
            \begin{enumerate}
                \item 发展生态农业,增施有机肥,少用农药化肥
                \item 实施污染达标排放
                \item 科学灌溉
                \item 合理规划垃圾堆放场
            \end{enumerate}
            \item 技术措施
            \begin{enumerate}
                \item 深耕改土
                \item 发展生物降解技术,如蚯蚓
            \end{enumerate}
        \end{enumerate}
    \end{enumerate}
\end{document}
